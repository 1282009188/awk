% vim: ts=8 sts=8 sw=4 et tw=75
\chapter{索引}
\label{chap:index}

\marginpar{205}
索引中的页码指的是英文原版的页码, 与本书页边标注的页码一致.

\begin{multicols}{2}

\setlength\parindent{0pt}
\small

\pdfbookmark[2]{符号}{symbol}\medskip\textbf{\large{符号}}

\hangindent=2pc \hangafter=1 \verb'&&' AND operator (``与'' 运算符) 10, 31, 37, 158

\hangindent=2pc \hangafter=1 \verb'%=' assignment operator (赋值运算符) 38

\hangindent=2pc \hangafter=1 \verb'*=' assignment operator (赋值运算符) 38

\hangindent=2pc \hangafter=1 \verb'+=' assignment operator (赋值运算符) 38

\hangindent=2pc \hangafter=1 \verb'-=' assignment operator (赋值运算符) 38

\hangindent=2pc \hangafter=1 \verb'/=' assignment operator (赋值运算符) 38

\hangindent=2pc \hangafter=1 \verb'^=' assignment operator (赋值运算符) 38

\hangindent=2pc \hangafter=1 \verb'=' assignment operator (赋值运算符) 38, 44

\hangindent=2pc \hangafter=1 \verb'\' backslash (反斜杠) 28, 30, 41, 43

\hangindent=2pc \hangafter=1 \verb'{'...\verb'}' braces (花括号) 15, 22, 47, 167, 188

\hangindent=2pc \hangafter=1 \verb'#' comment (注释) 15, 22, 188

\hangindent=2pc \hangafter=1 \verb'>' comparison operator (比较运算符) 9

\hangindent=2pc \hangafter=1 \verb'>=' comparison operator (比较运算符) 9

\hangindent=2pc \hangafter=1 \verb'==' comparison operator (比较运算符) 9, 44

\hangindent=2pc \hangafter=1 \verb'?:' conditional expression (条件表达式) 37, 68

\hangindent=2pc \hangafter=1 \verb',' continuation after (延续) 22, 188

\hangindent=2pc \hangafter=1 \verb'--' decrement operator (自减运算符) 39, 70, 112

\hangindent=2pc \hangafter=1 \verb'^' exponentiation operator (指数运算符) 15, 36, 46

\hangindent=2pc \hangafter=1 \verb'%' format conversion (格式转换) 42, 189

\hangindent=2pc \hangafter=1 \verb'-' in character class (字符类中的
\verb'-') 29

\hangindent=2pc \hangafter=1 \verb'%%' in \verb'printf' (\verb'printf' 中的
\verb'%%') 79

\hangindent=2pc \hangafter=1 \verb'&' in substitution (置换中的
\verb'&') 43, 72, 189

\hangindent=2pc \hangafter=1 \verb'++' increment operator (自增运算符) 39, 146, 158

\hangindent=2pc \hangafter=1 \verb'|' input pipe (输入管道) 62, 76

\hangindent=2pc \hangafter=1 \verb'~' match operator (匹配运算符) 25, 27, 31, 37, 40

\hangindent=2pc \hangafter=1 \verb'!~' nonmatch operator (不匹配运算符) 25, 27, 31, 37, 40

\hangindent=2pc \hangafter=1 \verb'!' NOT operator (``非'' 运算符) 10, 31, 37

\hangindent=2pc \hangafter=1 \verb'||' OR operator (``或'' 运算符) 10, 31, 37

\hangindent=2pc \hangafter=1 \verb'>' output redirection (输出重定向) 56, 58, 188

\hangindent=2pc \hangafter=1 \verb'>>' output redirection (输出重定向) 56, 58, 188

\hangindent=2pc \hangafter=1 \verb'|' output redirection (输出重定向) 58, 188

\hangindent=2pc \hangafter=1 \verb"' '" quotes (单引号) 2, 4-5, 65, 100

\hangindent=2pc \hangafter=1 \verb'.' regular expression (正则表达式) 28

\hangindent=2pc \hangafter=1 \verb'$' regular expression (正则表达式) 28

\hangindent=2pc \hangafter=1 \verb'^' regular expression (正则表达式) 28

\hangindent=2pc \hangafter=1 \verb'()' regular expression (正则表达式) 29

\hangindent=2pc \hangafter=1 \verb'['...\verb']' regular expression
(正则表达式) 29

\hangindent=2pc \hangafter=1 \verb'[^'...\verb']' regular expression
(正则表达式) 29

\hangindent=2pc \hangafter=1 \verb'|' regular expression (正则表达式) 29, 32

\hangindent=2pc \hangafter=1 \verb'*' regular expression (正则表达式) 30

\hangindent=2pc \hangafter=1 \verb'+' regular expression (正则表达式) 30

\hangindent=2pc \hangafter=1 \verb'?' regular expression (正则表达式) 30

\hangindent=2pc \hangafter=1 \verb'%' remainder operator (取余运算符) 36, 46

\hangindent=2pc \hangafter=1 \verb'-' standard input filename
(标准输入的文件名) 64, 116

\hangindent=2pc \hangafter=1 \verb'"'...\verb'"' string constant
(字符串常量) 7, 24, 35, 51

\hangindent=2pc \hangafter=1 \verb'_' underscore (下划线) 35

\hangindent=2pc \hangafter=1 \verb'$0' at end of input (输入结束时的
\verb'$0') 13

\hangindent=2pc \hangafter=1 \verb'$0' blank line (空行) 192

\hangindent=2pc \hangafter=1 \verb'$0' record variable (记录变量) 5, 35

\hangindent=2pc \hangafter=1 \verb'$0', side-effects on (\verb'$0' 上的副作用)
36 43

\hangindent=2pc \hangafter=1 \verb'\007' bell character (响铃符) 31

\hangindent=2pc \hangafter=1 \verb'\b' backspace character (退格符) 31

\hangindent=2pc \hangafter=1 \verb'/dev/tty' file (文件 \verb'/dev/tty') 59

\hangindent=2pc \hangafter=1 \verb'-f' option (选项 \verb'-f') 4-5, 63, 65, 187

\hangindent=2pc \hangafter=1 \verb'-F' option (选项 \verb'-F') 60, 63, 187

\hangindent=2pc \hangafter=1 \verb'>'\textit{file}, \verb'print' 90, 188

\hangindent=2pc \hangafter=1 \verb'#include' processor (\verb'#include'
预处理器) 62, 64

\hangindent=2pc \hangafter=1 \verb'$'\textit{n} field (\verb'$'\textit{n}
字段) 5, 35

\hangindent=2pc \hangafter=1 \verb'\n' newline character (换行符) 8, 31, 79

\hangindent=2pc \hangafter=1 $\pi$, computation of (计算 $\pi$) 39

\hangindent=2pc \hangafter=1 \verb'\t' tab character (制表符) 15, 24, 31

\hangindent=2pc \hangafter=1 \verb'$x++' versus \verb'$(x++)' (\verb'$x++'
与 \verb'$(x++)') 146

\pdfbookmark[2]{A}{A}\medskip\textbf{\large{A}}

\hangindent=2pc \hangafter=1 action, default (默认动作) 5, 9, 21, 187

\hangindent=2pc \hangafter=1 actions, summary of (动作的总结) 34, 188

\hangindent=2pc \hangafter=1 add checks and deposits (累加支出与存款) 87

\hangindent=2pc \hangafter=1 \verb'addcomma' program (程序 \verb'addcomma') 72, 194

\hangindent=2pc \hangafter=1 address list (地址列表) 82

\hangindent=2pc \hangafter=1 address list, sorting (排序地址列表) 84

\hangindent=2pc \hangafter=1 aggregation (聚合) 51, 58, 119

\hangindent=2pc \hangafter=1 Aho, A. V. 130, 152, 179, 186

\hangindent=2pc \hangafter=1 Aho, S. vi

\hangindent=2pc \hangafter=1 Akkerhuis, J. vi

\hangindent=2pc \hangafter=1 algorithm, depth-first search (深度优先搜索) 172, 177

\hangindent=2pc \hangafter=1 algorithm, heapsort (堆排序) 162

\hangindent=2pc \hangafter=1 algorithm, insertion sort (插入排序) 153

\hangindent=2pc \hangafter=1 algorithm, linear (线性算法) 157, 183

\hangindent=2pc \hangafter=1 algorithm, \verb'make' update (程序
\verb'make') 176

\hangindent=2pc \hangafter=1 algorithm, $n\log n$ (对数算法) 162, 165

\hangindent=2pc \hangafter=1 algorithm, quadratic (平方算法) 157, 162, 183

\hangindent=2pc \hangafter=1 algorithm, quicksort (快速排序) 160

\hangindent=2pc \hangafter=1 algorithm, topological sort (拓扑排序) 171

\hangindent=2pc \hangafter=1 AND operator, \verb'&&' (``与'' 运算符) 10, 31, 37, 158

\hangindent=2pc \hangafter=1 \verb'ARGC' variable (变量 \verb'ARGC') 36, 63, 189

\hangindent=2pc \hangafter=1 arguments, command-line (命令行参数) 63

\hangindent=2pc \hangafter=1 arguments, function (函数参数) 54

\hangindent=2pc \hangafter=1 \verb'ARGV', changing (修改 \verb'ARGV') 64-65, 116

\hangindent=2pc \hangafter=1 \verb'ARGV' variable (变量 \verb'ARGV') 36, 63-65, 116, 189

\hangindent=2pc \hangafter=1 \verb'arith' program (程序 \verb'arith') 117

\hangindent=2pc \hangafter=1 arithmetic expression grammar (算术表达式语法) 145

\hangindent=2pc \hangafter=1 arithmetic functions, summary of
(算术表达式总结) 190

\hangindent=2pc \hangafter=1 arithmetic functions, table of
(算术表达式总结, 表格) 39

\hangindent=2pc \hangafter=1 arithmetic operator (算术运算符) 36, 44

\hangindent=2pc \hangafter=1 arithmetic operator, table of (算术运算符总结,
表格) 46

\hangindent=2pc \hangafter=1 array, associative 50-51, (关联数组) 193

\hangindent=2pc \hangafter=1 array parameter (数组参数) 54

\hangindent=2pc \hangafter=1 array reference, cost of (引用数组的开销) 184, 204

\hangindent=2pc \hangafter=1 array subscripts (数组的下标) 50-52

\hangindent=2pc \hangafter=1 arrays (数组) 16, 50

\hangindent=2pc \hangafter=1 arrays, multidimensional (多维数组) 52, 108, 114, 116, 182

\hangindent=2pc \hangafter=1 \verb'asm' program (程序 \verb'asm') 134, 203

\hangindent=2pc \hangafter=1 \verb'asplit' function (函数 \verb'asplit') 81

\hangindent=2pc \hangafter=1 assembler instructions, table of
(汇编程序的指令, 表格) 132

\hangindent=2pc \hangafter=1 assembly language (汇编语言) 133

\hangindent=2pc \hangafter=1 assignment, command-line (在命令行赋值) 63, 94, 187, 195, 197

\hangindent=2pc \hangafter=1 assignment expression (赋值表达式) 39, 127

\hangindent=2pc \hangafter=1 assignment, multiple (多重赋值) 39

\hangindent=2pc \hangafter=1 assignment operator, \verb'%=' (赋值运算符) 38

\hangindent=2pc \hangafter=1 assignment operator, \verb'*=' (赋值运算符) 38

\hangindent=2pc \hangafter=1 assignment operator, \verb'+=' (赋值运算符) 38

\hangindent=2pc \hangafter=1 assignment operator, \verb'-=' (赋值运算符) 38

\hangindent=2pc \hangafter=1 assignment operator, \verb'/=' (赋值运算符) 38

\hangindent=2pc \hangafter=1 assignment operator, \verb'^=' (赋值运算符) 38

\hangindent=2pc \hangafter=1 assignment operator, \verb'=' (赋值运算符) 38, 44

\hangindent=2pc \hangafter=1 assignment operators (赋值运算符) 38

\hangindent=2pc \hangafter=1 assignment, side-effects of (赋值的副作用) 43

\hangindent=2pc \hangafter=1 associative array (关联数组) 50-51, 193

\hangindent=2pc \hangafter=1 associative of operators (运算符之间的关联) 46

\hangindent=2pc \hangafter=1 \verb'atan2' function (函数 \verb'atan2') 39

\hangindent=2pc \hangafter=1 attribute, database (数据库中的属性) 103

\hangindent=2pc \hangafter=1 avoiding \verb'sort' options (避开 \verb'sort'
的选项) 91, 140

\hangindent=2pc \hangafter=1 awk command line (awk 的命令行) 1, 3, 63, 65, 187

\hangindent=2pc \hangafter=1 awk grammar (awk 的语法) 148

\hangindent=2pc \hangafter=1 awk program, form of (awk 程序的形式) 2, 21, 187

\hangindent=2pc \hangafter=1 awk program, running an (运行 awk 程序) 3

\hangindent=2pc \hangafter=1 awk program, running time of (awk
程序的运行时间) 183

\hangindent=2pc \hangafter=1 \verb'awk.parser' program (程序
\verb'awk.parser') 149

\pdfbookmark[2]{B}{B}\medskip\textbf{\large{B}}

\hangindent=2pc \hangafter=1 back edge (回边) 173-174

\hangindent=2pc \hangafter=1 backslash, \verb'\' (反斜杠) 28, 30, 41, 43

\hangindent=2pc \hangafter=1 backspace character, \verb'\b' (退格符) 31

\hangindent=2pc \hangafter=1 bailing out (紧急退出) 4

\hangindent=2pc \hangafter=1 balanced delimiters (均衡的分隔符) 77, 195

\hangindent=2pc \hangafter=1 base table (基表) 106

\hangindent=2pc \hangafter=1 batch sort test program (批处理形式的排序测
试程序) 155

\hangindent=2pc \hangafter=1 \verb'BEGIN' and \verb'END', multiple (多重
\verb'BEGIN' 与 \verb'END') 23, 169

\hangindent=2pc \hangafter=1 \verb'BEGIN' pattern (模式 \verb'BEGIN') 11, 23, 63

\hangindent=2pc \hangafter=1 bell character, \verb'\007' (响铃符) 31

\hangindent=2pc \hangafter=1 Bentley, J. L. vi, 130, 152, 179
% \hangindent=2pc \hangafter=1 \marginpar{206}

\hangindent=2pc \hangafter=1 binary tree (二叉树) 163

\hangindent=2pc \hangafter=1 blank line, \verb'$0' (空行) 192

\hangindent=2pc \hangafter=1 blank line, printing a (打印一个空行) 11, 55

\hangindent=2pc \hangafter=1 blank line separator (空行分隔符) 83

\hangindent=2pc \hangafter=1 boundary condition testing (边界条件测试) 155

\hangindent=2pc \hangafter=1 braces, \verb'{'...\verb'}' (花括号) 15,22,47, 167, 188

\hangindent=2pc \hangafter=1 breadth-first order (宽度优先顺序) 163, 171

\hangindent=2pc \hangafter=1 \verb'break' statement (\verb'break' 语句) 49

\hangindent=2pc \hangafter=1 \verb'bridge' program (程序 \verb'bridge') 199

\hangindent=2pc \hangafter=1 built-in variables, table of (内建变量总结, 
表格) 36

\hangindent=2pc \hangafter=1 \verb'bundle' program (程序 \verb'bundle') 81

\pdfbookmark[2]{C}{C}\medskip\textbf{\large{C}}

\hangindent=2pc \hangafter=1 \verb'calc1' program (程序 \verb'calc1') 143

\hangindent=2pc \hangafter=1 \verb'calc2' program (程序 \verb'calc2') 144

\hangindent=2pc \hangafter=1 \verb'calc3' program (程序 \verb'calc3') 146

\hangindent=2pc \hangafter=1 call by reference (引用传递) 54

\hangindent=2pc \hangafter=1 call by value (值传递) 54

\hangindent=2pc \hangafter=1 \verb'capitals' file (文件 \verb'capitals') 102

\hangindent=2pc \hangafter=1 \verb'cat' command (命令 \verb'cat') 59, 64

\hangindent=2pc \hangafter=1 \verb'cc' command (命令 \verb'cc') 175

\hangindent=2pc \hangafter=1 changing \verb'ARGV' (修改 \verb'ARGV') 64-65, 116

\hangindent=2pc \hangafter=1 character class, \verb'-' in (字符类) 29

\hangindent=2pc \hangafter=1 character class, complemented (互补字符类) 29

\hangindent=2pc \hangafter=1 character class, regular expression
(正则表达式中的字符类) 29

\hangindent=2pc \hangafter=1 characters, table of escape (转义字符总结,
表格) 31

\hangindent=2pc \hangafter=1 \verb'check' function (函数 \verb'check') 155

\hangindent=2pc \hangafter=1 check password file (检查密码文件) 78

\hangindent=2pc \hangafter=1 \verb'check1' program (程序 \verb'check1') 87

\hangindent=2pc \hangafter=1 \verb'check2' program (程序 \verb'check2') 87

\hangindent=2pc \hangafter=1 \verb'check3' program (程序 \verb'check3') 88

\hangindent=2pc \hangafter=1 \verb'checkgen' program (程序 \verb'checkgen') 79

\hangindent=2pc \hangafter=1 checking, cross-reference (交叉引用检查) 73

\hangindent=2pc \hangafter=1 checks and deposits, add (累加支出与存款) 87

\hangindent=2pc \hangafter=1 checks, printing (打印支票) 74

\hangindent=2pc \hangafter=1 Cherry, L. L. vi

\hangindent=2pc \hangafter=1 \verb'chmod' command (命令 \verb'chmod') 65

\hangindent=2pc \hangafter=1 \verb'choose' function (命令 \verb'choose') 112

\hangindent=2pc \hangafter=1 \verb'cliche' program (程序 \verb'cliche') 113

\hangindent=2pc \hangafter=1 \verb'close' statement (\verb'close' 语句) 59, 82

\hangindent=2pc \hangafter=1 coercion (强制类型转换) 44, 154, 182

\hangindent=2pc \hangafter=1 coercion, number to string
(数值强制转换成字符串) 25

\hangindent=2pc \hangafter=1 coercion rules (强制类型转换的规则) 44, 192

\hangindent=2pc \hangafter=1 coercion, string to number
(字符串强制转换成数值) 25

\hangindent=2pc \hangafter=1 coercion to number (强制转换成数值) 45

\hangindent=2pc \hangafter=1 coercion to string (强制转换成字符串) 45

\hangindent=2pc \hangafter=1 \verb'colcheck' program (程序 \verb'colcheck') 77

\hangindent=2pc \hangafter=1 columns, summing (列求和) 67

\hangindent=2pc \hangafter=1 command, \verb'cat' (命令 \verb'cat') 59, 64

\hangindent=2pc \hangafter=1 command, \verb'cc' (命令 \verb'cc') 175

\hangindent=2pc \hangafter=1 command, \verb'chmod' (命令 \verb'chmod') 65

\hangindent=2pc \hangafter=1 command, \verb'date' (命令 \verb'date') 62, 76

\hangindent=2pc \hangafter=1 command, \verb'egrep' (命令 \verb'egrep')
59, 181, 184,  186

\hangindent=2pc \hangafter=1 command, \verb'grep' (命令 \verb'grep') v, 181, 184

\hangindent=2pc \hangafter=1 command interpreter, shell (命令解释器 shell)
4, 65,  99

\hangindent=2pc \hangafter=1 command, \verb'join' (命令 \verb'join') 104

\hangindent=2pc \hangafter=1 command line, \verb'awk' (awk 的命令行) 1, 3, 63, 65, 187

\hangindent=2pc \hangafter=1 command, \verb'lorder' (命令 \verb'lorder') 170

\hangindent=2pc \hangafter=1 command, \verb'ls' (命令 \verb'ls') 177

\hangindent=2pc \hangafter=1 command, \verb'make' (命令 \verb'make') 175

\hangindent=2pc \hangafter=1 command, \verb'nm' (命令 \verb'nm') 73

\hangindent=2pc \hangafter=1 command, \verb'pr' (命令 \verb'pr') 175

\hangindent=2pc \hangafter=1 command, \verb'ptx' (命令 \verb'ptx') 123

\hangindent=2pc \hangafter=1 command, \verb'sed' (命令 \verb'sed') 
v, 181, 184, 186

\hangindent=2pc \hangafter=1 command, \verb'sort' (命令 \verb'sort') 8, 58, 84, 90

\hangindent=2pc \hangafter=1 command, \verb'tbl' (命令 \verb'tbl') 95

\hangindent=2pc \hangafter=1 command, \verb'tr' (命令 \verb'tr') 201

\hangindent=2pc \hangafter=1 command, \verb'troff' (命令 \verb'troff')
120, 124-125, 127, 139

\hangindent=2pc \hangafter=1 command, \verb'wc' (命令 \verb'wc') 183


\hangindent=2pc \hangafter=1 command, \verb'who' (命令 \verb'who') 62

\hangindent=2pc \hangafter=1 command-line arguments (命令行参数) 63

\hangindent=2pc \hangafter=1 command-line assignment (命令行赋值)
63, 94, 187, 195, 197

\hangindent=2pc \hangafter=1 commas, inserting (插入逗号) 71

\hangindent=2pc \hangafter=1 comment, \verb'#' (注释) 15, 22, 188

\hangindent=2pc \hangafter=1 comparison expression, value of
(比较表达式的值) 37

\hangindent=2pc \hangafter=1 comparison, numeric (数值比较) 25-26, 44

\hangindent=2pc \hangafter=1 comparison operator, \verb'>' (比较运算符) 9

\hangindent=2pc \hangafter=1 comparison operator, \verb'>=' (比较运算符) 9

\hangindent=2pc \hangafter=1 comparison operator, \verb'==' (比较运算符) 9, 44

\hangindent=2pc \hangafter=1 comparison operators (比较运算符) 36

\hangindent=2pc \hangafter=1 comparison operators, table of (比较运算符总结,
 表格) 25

\hangindent=2pc \hangafter=1 comparison, string (字符串的比较) 25-26, 44, 184

\hangindent=2pc \hangafter=1 \verb'compat' program (程序 \verb'compat') 80

\hangindent=2pc \hangafter=1 compiler model (编译器模型) 131

\hangindent=2pc \hangafter=1 complemented character class (互补字符类) 29

\hangindent=2pc \hangafter=1 compound patterns (复合模式) 31

\hangindent=2pc \hangafter=1 computation of base-10 logarithm (计算以 10
为底的对数) 39

\hangindent=2pc \hangafter=1 computation of $e$ (计算 $e$) 39

\hangindent=2pc \hangafter=1 computation of $\pi$ (计算 $\pi$) 39

\hangindent=2pc \hangafter=1 concatenation in regular expression
(正则表达式中的拼接) 29

\hangindent=2pc \hangafter=1 concatenation operator (拼接运算符) 40, 43, 182

\hangindent=2pc \hangafter=1 concatenation, string (字符串拼接) 13,40,43, 47, 56, 101, 182, 184

\hangindent=2pc \hangafter=1 conditional expression, \verb'?:' (条件表达式) 37, 68

\hangindent=2pc \hangafter=1 constant, \verb'"'...\verb'"' string
(字符串常量) 7, 24, 35, 51

\hangindent=2pc \hangafter=1 constant, numeric (数值常量) 35

\hangindent=2pc \hangafter=1 constraint graph (约束关系图) 170

\hangindent=2pc \hangafter=1 context-free grammar (上下文无关语法) 113,145, 148

\hangindent=2pc \hangafter=1 continuation after \verb',' (逗号后的延续) 22, 188

\hangindent=2pc \hangafter=1 \verb'continue' statement (\verb'continue'
语句) 49

\hangindent=2pc \hangafter=1 continuing long statements (延续长语句) 15, 22, 188

\hangindent=2pc \hangafter=1 control-break program (control-break 程序) 92, 95, 105, 110,126

\hangindent=2pc \hangafter=1 control-flow statements, summary of
(流程控制语句的总结) 48

\hangindent=2pc \hangafter=1 conversion, \verb'%' format (格式转换) 42, 189

\hangindent=2pc \hangafter=1 conversion, date (日期的转换) 72, 194

\hangindent=2pc \hangafter=1 conversion, number to string
(数值到字符串的转换) 35, 44, 192

\hangindent=2pc \hangafter=1 conversion, string to number
(字符串到数值的转换) 35, 44, 192

\hangindent=2pc \hangafter=1 convert numbers to words (数值转换成单词) 76

\hangindent=2pc \hangafter=1 \verb'cos' function (函数 \verb'cos') 39

\hangindent=2pc \hangafter=1 cost of array reference (数组引用的开销) 184, 204

\hangindent=2pc \hangafter=1 \verb'countries' file (文件 \verb'countries') 22

\hangindent=2pc \hangafter=1 Cowlishaw, M. F. 186

\hangindent=2pc \hangafter=1 cross-reference checking (交叉引用检查) 73

\hangindent=2pc \hangafter=1 cross-references in manuscripts
(手稿的交叉引用) 120

\hangindent=2pc \hangafter=1 cycle, graph (图中的环) 171, 173-174, 177

\pdfbookmark[2]{D}{D}\medskip\textbf{\large{D}}

\hangindent=2pc \hangafter=1 Dallen, J. A. 152

\hangindent=2pc \hangafter=1 data, name-value (\mbox{名字}-值) 86

\hangindent=2pc \hangafter=1 data, regular expressions in
(带有正则表达式的数据) 118

\hangindent=2pc \hangafter=1 data, self-identifying (自描述数据) 86

\hangindent=2pc \hangafter=1 data structure, successor-list (后继节点列表) 171

\hangindent=2pc \hangafter=1 data types (数据类型) 5

\hangindent=2pc \hangafter=1 data validation (数据验证) 10, 76

\hangindent=2pc \hangafter=1 database attribute (数据库中的属性) 103

\hangindent=2pc \hangafter=1 database description (数据库描述文件) ,
\verb'relfile' 106

\hangindent=2pc \hangafter=1 database, multifile (多文件数据库) 102

\hangindent=2pc \hangafter=1 database query (数据库查询) 99

\hangindent=2pc \hangafter=1 database table (数据库表格) 103

\hangindent=2pc \hangafter=1 databases, relational (关系数据库) iv, 102


\hangindent=2pc \hangafter=1 \verb'date' command (命令 \verb'date') 62, 76

\hangindent=2pc \hangafter=1 date conversion (数据转换) 72, 194

\hangindent=2pc \hangafter=1 dates, sorting (日期排序) 72

\hangindent=2pc \hangafter=1 \verb'daynum' function (函数 \verb'daynum') 194

\hangindent=2pc \hangafter=1 decrement operator, \verb'--' (自减运算符) 39, 70, 112

\hangindent=2pc \hangafter=1 default action (默认动作) 5, 9, 21, 187

\hangindent=2pc \hangafter=1 default field separator (默认的字段分隔符) 5, 24

\hangindent=2pc \hangafter=1 default initialization (默认的初始化) 12-13, 35, 38,45,50-51,54,68, 181

\hangindent=2pc \hangafter=1 \verb'delete' statement (\verb'delete' 语句) 52

\hangindent=2pc \hangafter=1 delimiters, balanced (均衡的分隔符) 77, 195

\hangindent=2pc \hangafter=1 dependency description, \verb'makefile'
(\verb'makefile' 的依赖关系描述) 175

\hangindent=2pc \hangafter=1 dependency graph (依赖关系图) 176

\hangindent=2pc \hangafter=1 depth-first search algorithm (深度优先搜索) 172, 177

\hangindent=2pc \hangafter=1 derived table (导出表) 106

\hangindent=2pc \hangafter=1 \verb'dfs' function (函数 \verb'dfs') 173

\hangindent=2pc \hangafter=1 divide and conquer (分而治之) v, 89, 110, 121, 123-124, 130, 160, 184, 202

\hangindent=2pc \hangafter=1 \verb'do' statement (\verb'do' 语句) 49

\hangindent=2pc \hangafter=1 dynamic regular expression (动态的正则表达式) 40,  101, 184

\pdfbookmark[2]{E}{E}\medskip\textbf{\large{E}}

\hangindent=2pc \hangafter=1 $e$, computation of (计算 $e$) 39

\hangindent=2pc \hangafter=1 \verb'echo' program (程序 \verb'echo') 63

\hangindent=2pc \hangafter=1 \verb'egrep' command (命令 \verb'egrep') 59,181,184, 186

\hangindent=2pc \hangafter=1 \verb'else', semicolon before (\verb'else'
前面的分号) 47

\hangindent=2pc \hangafter=1 \verb'emp.data' file (文件 \verb'emp.data') 1

\hangindent=2pc \hangafter=1 empty statement (空语句) 50,188

\hangindent=2pc \hangafter=1 \verb'END', multiple \verb'BEGIN' and (多个
\verb'END' 与 \verb'BEGIN') 23, 169

\hangindent=2pc \hangafter=1 end of input, \verb'$0' at (输入结束时的
\verb'$0') 13

\hangindent=2pc \hangafter=1 \verb'END' pattern (\verb'END' 模式) 11, 23, 49

\hangindent=2pc \hangafter=1 error file, standard (标准错误文件) 59

\hangindent=2pc \hangafter=1 \verb'error' function (函数 \verb'error')
118, 149, 178

\hangindent=2pc \hangafter=1 error messages, printing (打印错误信息) 59

\hangindent=2pc \hangafter=1 error, syntax (语法错误) 4

\hangindent=2pc \hangafter=1 escape sequence (转义序列) 31, 35, 191

\hangindent=2pc \hangafter=1 escape sequences, table of (转义序列总结,
表格) 31

\hangindent=2pc \hangafter=1 examples, regular expression (正则表达式的例子) 30

\hangindent=2pc \hangafter=1 examples, table of \verb'printf'
(\verb'printf' 的例子) 57

\hangindent=2pc \hangafter=1 executable file (可执行文件) 65

\hangindent=2pc \hangafter=1 \verb'exit' statement (\verb'exit' 语句) 49

\hangindent=2pc \hangafter=1 exit status (退出状态) 50, 64

\hangindent=2pc \hangafter=1 \verb'exp' function (函数 \verb'exp') 39

\hangindent=2pc \hangafter=1 exponential notation (指数的记号) 35

\hangindent=2pc \hangafter=1 exponentiation operator, \verb'^' (指数运算符)
15, 36, 46

\hangindent=2pc \hangafter=1 expression, \verb'?:' conditional (条件表达式)
37, 68

\hangindent=2pc \hangafter=1 expression, assignment (赋值表达式) 39, 127

\hangindent=2pc \hangafter=1 expression grammar (表达式语法) 145

\hangindent=2pc \hangafter=1 expression, value of comparison (比较表达式的值) 37

\hangindent=2pc \hangafter=1 expression, value of logical (逻辑表达式的值) 37

\hangindent=2pc \hangafter=1 expressions, field (字段) 36

\hangindent=2pc \hangafter=1 expressions, primary (初等表达式) 34

\hangindent=2pc \hangafter=1 expressions, summary of (表达式的总结) 37

\pdfbookmark[2]{F}{F}\medskip\textbf{\large{F}}

\hangindent=2pc \hangafter=1 Feldman,S. I. 179

\hangindent=2pc \hangafter=1 field expressions (字段表达式) 36

\hangindent=2pc \hangafter=1 field, \verb'$'$n$ (字段 \verb'$'$n$) 5, 35

\hangindent=2pc \hangafter=1 field, nonexistent (不存在的字段) 36,45, 192

\hangindent=2pc \hangafter=1 \verb'field' program (程序 \verb'field') 66

\hangindent=2pc \hangafter=1 field separator, default (默认的字段分隔符) 5, 24

\hangindent=2pc \hangafter=1 field separator, input (输入数据的字段分隔符)
24, 35, 39, 60

\hangindent=2pc \hangafter=1 field separator, newline as
(换行符作为字段分隔符) 61, 83-84

\hangindent=2pc \hangafter=1 field separator, output (输出数据的字段分隔符)
6, 35, 39, 55-56

% \hangindent=2pc \hangafter=1 \marginpar{207}
\hangindent=2pc \hangafter=1 field separator, regular expression
(用正则表达式作字段分隔符) 52, 60, 80, 135

\hangindent=2pc \hangafter=1 field variables (字段变量) 35

\hangindent=2pc \hangafter=1 fields, named (命名字段) 102, 107

\hangindent=2pc \hangafter=1 file, \verb'capitals' (文件 \verb'capitals') 102

\hangindent=2pc \hangafter=1 file, \verb'countries' (文件 \verb'countries') 22

\hangindent=2pc \hangafter=1 file, \verb'/dev/tty' (文件 \verb'/dev/tty') 59

\hangindent=2pc \hangafter=1 file, \verb'emp.data' (文件 \verb'emp.data') 1

\hangindent=2pc \hangafter=1 file, executable (可执行文件) 65

\hangindent=2pc \hangafter=1 file, standard error (标准错误文件) 59

\hangindent=2pc \hangafter=1 file, standard input (标准输入文件) 59, 66

\hangindent=2pc \hangafter=1 file, standard output (标准输出文件) 5, 56

\hangindent=2pc \hangafter=1 file updating (文件更新) 175

\hangindent=2pc \hangafter=1 filename, \verb'-' standard input
(标准输入的文件名) 64, 116

\hangindent=2pc \hangafter=1 \verb'FILENAME' variable (变量
\verb'FILENAME') 33, 35-36, 81, 103

\hangindent=2pc \hangafter=1 fixed-field input (字段固定的输入) 72

\hangindent=2pc \hangafter=1 floating-point number, regular expression for
(浮点数的正则表达式) 30, 40

\hangindent=2pc \hangafter=1 floating-point precision (浮点数精度) 35, 191

\hangindent=2pc \hangafter=1 Floyd, R. W. 162, 198

\hangindent=2pc \hangafter=1 \verb'fmt' program (程序 \verb'fmt') 120

\hangindent=2pc \hangafter=1 \verb'fmt.just' program (程序 \verb'fmt.just') 202

\hangindent=2pc \hangafter=1 \verb'FNR' variable (变量 \verb'FNR') 33, 35-36, 61

\hangindent=2pc \hangafter=1 \verb'for' ... \verb'in' statement (\verb'for'
... \verb'in' 语句) 51

\hangindent=2pc \hangafter=1 \verb'for(;;)' infinite loop (无限循环) 49, 113

\hangindent=2pc \hangafter=1 \verb'for' statement (\verb'for' 语句) 16, 49

\hangindent=2pc \hangafter=1 form letters (格式信函) 100

\hangindent=2pc \hangafter=1 form of awk program (awk 程序的形式) 2, 21, 187

\hangindent=2pc \hangafter=1 \verb'form1' program (程序 \verb'form1') 91

\hangindent=2pc \hangafter=1 \verb'form2' program (程序 \verb'form2') 92

\hangindent=2pc \hangafter=1 \verb'form3' program (程序 \verb'form3') 94

\hangindent=2pc \hangafter=1 \verb'form4' program (程序 \verb'form4') 96

\hangindent=2pc \hangafter=1 formal parameters (形式参数) 54

\hangindent=2pc \hangafter=1 format, program (程序的格式)
11, 22, 34, 47, 53, 188

\hangindent=2pc \hangafter=1 formatting, table (表格的格式化) 95

\hangindent=2pc \hangafter=1 \verb'form.gen' program (程序 \verb'form.gen') 101

\hangindent=2pc \hangafter=1 Forth language (Forth 编程语言) 142

\hangindent=2pc \hangafter=1 Fraser, C. W. vi

\hangindent=2pc \hangafter=1 \verb'FS' variable (变量 \verb'FS')
24, 35-36, 52, 60, 83, 135, 187

\hangindent=2pc \hangafter=1 function arguments (函数参数) 54

\hangindent=2pc \hangafter=1 function, \verb'asplit' (函数 \verb'asplit') 81

\hangindent=2pc \hangafter=1 function, \verb'atan2' (函数 \verb'atan2') 39

\hangindent=2pc \hangafter=1 function, \verb'check' (函数 \verb'check') 155

\hangindent=2pc \hangafter=1 function, \verb'choose' (函数 \verb'choose') 112

\hangindent=2pc \hangafter=1 function, \verb'cos' (函数 \verb'cos') 39

\hangindent=2pc \hangafter=1 function, \verb'daynum' (函数 \verb'daynum') 194

\hangindent=2pc \hangafter=1 function definition (函数定义) 53, 187

\hangindent=2pc \hangafter=1 function, \verb'dfs' (函数 \verb'dfs') 173

\hangindent=2pc \hangafter=1 function, \verb'error' (函数 \verb'error')
118, 149, 178

\hangindent=2pc \hangafter=1 function, \verb'exp' (函数 \verb'exp') 39

\hangindent=2pc \hangafter=1 function, \verb'getline' 61, 182, 188

\hangindent=2pc \hangafter=1 function, \verb'gsub' (函数 \verb'gsub')
42, 71, 101, 119, 122, 182

\hangindent=2pc \hangafter=1 function, \verb'heapify' (函数 \verb'heapify')
163, 165

\hangindent=2pc \hangafter=1 function, \verb'hsort' (函数 \verb'hsort') 165

\hangindent=2pc \hangafter=1 function, \verb'index' (函数 \verb'index') 41

\hangindent=2pc \hangafter=1 function, \verb'int' (函数 \verb'int') 39-40

\hangindent=2pc \hangafter=1 function, \verb'isort' (函数 \verb'isort') 154

\hangindent=2pc \hangafter=1 function, \verb'length' (函数 \verb'length') 13-14

\hangindent=2pc \hangafter=1 function, \verb'log' (函数 \verb'log') 39

\hangindent=2pc \hangafter=1 function, \verb'match' (函数 \verb'match')
35, 41, 149, 182, 189, 196

\hangindent=2pc \hangafter=1 function, \verb'max' (函数 \verb'max') 53

\hangindent=2pc \hangafter=1 function, \verb'numtowords' (函数
\verb'numtowords') 76, 194

\hangindent=2pc \hangafter=1 function, \verb'permute' (函数 \verb'permute') 199

\hangindent=2pc \hangafter=1 function, \verb'prefix' (函数 \verb'prefix') 105

\hangindent=2pc \hangafter=1 function, \verb'qsort' (函数 \verb'qsort') 161

\hangindent=2pc \hangafter=1 function, \verb'rand' (函数 \verb'rand') 40, 111

\hangindent=2pc \hangafter=1 function, \verb'randint' (函数 \verb'randint') 111

\hangindent=2pc \hangafter=1 function, \verb'randlet' (函数 \verb'randlet') 112

\hangindent=2pc \hangafter=1 function, \verb'recursive' (函数 \verb'recursive')
54, 71, 76, 115, 160

\hangindent=2pc \hangafter=1 function, \verb'sin' (函数 \verb'sin') 39

\hangindent=2pc \hangafter=1 function, \verb'split' (函数 \verb'split')
41, 52-53, 76, 80, 84, 192

\hangindent=2pc \hangafter=1 function, \verb'sprintf' (函数 \verb'sprintf')
42, 76, 88, 189

\hangindent=2pc \hangafter=1 function, \verb'sqrt' (函数 \verb'sqrt') 39

\hangindent=2pc \hangafter=1 function, \verb'srand' (函数 \verb'srand') 40, 111

\hangindent=2pc \hangafter=1 function, \verb'sub' (函数 \verb'sub') 42, 182

\hangindent=2pc \hangafter=1 function, \verb'subset' (函数 \verb'subset') 109

\hangindent=2pc \hangafter=1 function, \verb'substr' (函数 \verb'substr') 43, 72

\hangindent=2pc \hangafter=1 function, \verb'suffix' (函数 \verb'suffix') 105

\hangindent=2pc \hangafter=1 function, \verb'system' (函数 \verb'system') 59, 64

\hangindent=2pc \hangafter=1 function, \verb'unget' (函数 \verb'unget') 105

\hangindent=2pc \hangafter=1 function with counters, \verb'isort'
(带有计数语句的 \verb'isort') 158

\hangindent=2pc \hangafter=1 function, summary of arithmetic (算术函数总结) 190

\hangindent=2pc \hangafter=1 function, summary of string (字符串函数总结) 190

\hangindent=2pc \hangafter=1 function, table of arithmetic (算术函数表格) 39

\hangindent=2pc \hangafter=1 function, table of string (字符串函数表格) 42

\hangindent=2pc \hangafter=1 function, user-defined (用户自定义函数) 53, 182, 187

\pdfbookmark[2]{G}{G}\medskip\textbf{\large{G}}

\hangindent=2pc \hangafter=1 generation, program (程序的生成) v, 79, 121, 167

\hangindent=2pc \hangafter=1 generator, \verb'lex' lexical analyzer
(\verb'lex' 词法分析器) 152, 181, 186

\hangindent=2pc \hangafter=1 generator, \verb'yacc' parser (\verb'yacc'
语法分析器) 152, 175

\hangindent=2pc \hangafter=1 \verb'getline' error return (\verb'getline'
的错误返回值) 61-62

\hangindent=2pc \hangafter=1 \verb'getline' forms, table of (\verb'getline'
的形式) 62

\hangindent=2pc \hangafter=1 \verb'getline' function (函数 \verb'getline')
61, 182, 188

\hangindent=2pc \hangafter=1 \verb'getline', side-effects of
(\verb'getline' 的副作用) 61

\hangindent=2pc \hangafter=1 global variables (全局变量) 54, 116

\hangindent=2pc \hangafter=1 grammar, arithmetic expression
(算术表达式的语法) 145

\hangindent=2pc \hangafter=1 grammar, awk (awk 的语法) 148

\hangindent=2pc \hangafter=1 grammar, contex-free (上下文无关语法) 113, 145, 148

\hangindent=2pc \hangafter=1 \verb'grap' language (\verb'grap' 语言) 139, 152

\hangindent=2pc \hangafter=1 graph, constraint (约束关系图) 170

\hangindent=2pc \hangafter=1 graph cycle (图中的环) 171, 173-174, 177

\hangindent=2pc \hangafter=1 graph, dependency (依赖图) 176

\hangindent=2pc \hangafter=1 \verb'graph' language (\verb'graph' 语言) 135

\hangindent=2pc \hangafter=1 \verb'graph' program (程序 \verb'graph') 137

\hangindent=2pc \hangafter=1 \verb'grep' command (命令 \verb'grep') v, 181, 184

\hangindent=2pc \hangafter=1 Griswold, M. 186

\hangindent=2pc \hangafter=1 Griswold, R. 186

\hangindent=2pc \hangafter=1 Grosse, E. H. vi

\hangindent=2pc \hangafter=1 \verb'gsub' function (函数 \verb'gsub')
42, 71, 101, 119, 122, 182

\hangindent=2pc \hangafter=1 Gusella, R. vi

% 应该是个笔误, 因此把它注释掉
% \hangindent=2pc \hangafter=1 happiness 30

\pdfbookmark[2]{H}{H}\medskip\textbf{\large{H}}

\hangindent=2pc \hangafter=1 Hardin, R. H. 130

\hangindent=2pc \hangafter=1 headers, records with (带有头部的记录) 85

\hangindent=2pc \hangafter=1 \verb'heapify' function (函数 \verb'heapify') 163, 165

\hangindent=2pc \hangafter=1 heapsort algorithm (堆排序算法) 162

\hangindent=2pc \hangafter=1 heapsort performance (堆排序的性能) 165

\hangindent=2pc \hangafter=1 heapsort, profiling (堆排序的剖析) 168-169

\hangindent=2pc \hangafter=1 Herbst, R. T. vi

\hangindent=2pc \hangafter=1 \verb'histogram' program (程序
\verb'histogram') 70, 193

\hangindent=2pc \hangafter=1 Hoare, C. A. R. 160

\hangindent=2pc \hangafter=1 Hopcroft, J. E. 179

\hangindent=2pc \hangafter=1 \verb'hsort' function (函数 \verb'hsort') 165

\pdfbookmark[2]{I}{I}\medskip\textbf{\large{I}}

\hangindent=2pc \hangafter=1 ICON language (ICON 语言) 186

\hangindent=2pc \hangafter=1 \verb'if-else' statement (\verb'if-else' 语句) 14, 47

\hangindent=2pc \hangafter=1 implementation limits (实现的限制) 59, 61-62, 191

\hangindent=2pc \hangafter=1 \verb'in' operator (运算符 \verb'in') 52, 192

\hangindent=2pc \hangafter=1 increment operator, \verb'++' (自增运算符) 39, 146, 158

\hangindent=2pc \hangafter=1 \verb'index' function (函数 \verb'index') 41

\hangindent=2pc \hangafter=1 index, KWIC (KWIC 索引) 122

\hangindent=2pc \hangafter=1 indexing (索引) 124

\hangindent=2pc \hangafter=1 indexing pipeline (索引流水线) 129

\hangindent=2pc \hangafter=1 infinite loop, \verb'for(;;)' (无限循环) 49, 113

\hangindent=2pc \hangafter=1 infix notation (中缀表示法) 142, 145

\hangindent=2pc \hangafter=1 \verb'info' program (程序 \verb'info') 100

\hangindent=2pc \hangafter=1 initialization, default (默认的初始化)
12-13, 35, 38, 45, 50-51, 54, 68, 181

\hangindent=2pc \hangafter=1 initializing \verb'rand' (初始化 \verb'rand') 111

\hangindent=2pc \hangafter=1 input field separator (输入字段分隔符) 24, 35, 39, 60

\hangindent=2pc \hangafter=1 input file, standard (标准输入文件) 59, 66

\hangindent=2pc \hangafter=1 input filename, \verb'-' standard
(标准输入的文件名) 64, 116

\hangindent=2pc \hangafter=1 input, fixed-field (字段固定的输入) 72

\hangindent=2pc \hangafter=1 input line \verb'$0' (输入行的 \verb'$0') 5

\hangindent=2pc \hangafter=1 input pipe, \verb'|' (输入管道) 62, 76

\hangindent=2pc \hangafter=1 input pushback (输入撤回) 105, 110

\hangindent=2pc \hangafter=1 input, side-effects of (输入的副作用) 35

\hangindent=2pc \hangafter=1 input-output, summary of (输入输出总结) 188

\hangindent=2pc \hangafter=1 inserting commas (插入逗号) 71

\hangindent=2pc \hangafter=1 insertion sort algorithm (插入排序算法) 153

\hangindent=2pc \hangafter=1 insertion sort performance (插入排序的性能) 158

\hangindent=2pc \hangafter=1 \verb'int' function (函数 \verb'int') 39-40

\hangindent=2pc \hangafter=1 integer, rounding to nearest (取整) 40

\hangindent=2pc \hangafter=1 interactive test program (交互式的测试程序) 157

\hangindent=2pc \hangafter=1 interactive testing (交互式的测试) 156

\hangindent=2pc \hangafter=1 \verb'interest' program (程序 \verb'interest') 15 

\hangindent=2pc \hangafter=1 \verb'isort' function (函数 \verb'isort') 154

\hangindent=2pc \hangafter=1 \verb'isort' function with counters
(带计数功能的 \verb'isort' 函数) 158

\hangindent=2pc \hangafter=1 \verb'ix.collapse' program (程序
\verb'ix.collapse') 126

\hangindent=2pc \hangafter=1 \verb'ix.format' program (程序
\verb'ix.format') 129 

\hangindent=2pc \hangafter=1 \verb'ix.genkey' program (程序
\verb'ix.genkey') 128 

\hangindent=2pc \hangafter=1 \verb'ix.rotate' program (程序
\verb'ix.rotate') 127 

\hangindent=2pc \hangafter=1 \verb'ix.sort1' program (程序 \verb'ix.sort1') 126

\hangindent=2pc \hangafter=1 \verb'ix.sort2' program (程序 \verb'ix.sort2') 128 

\pdfbookmark[2]{J}{J}\medskip\textbf{\large{J}}

\hangindent=2pc \hangafter=1 \verb'join' command (命令 \verb'join') 104

\hangindent=2pc \hangafter=1 join, natural (自然连接) 103

\hangindent=2pc \hangafter=1 \verb'join' program (程序 \verb'join') 104

\hangindent=2pc \hangafter=1 justification, test (对齐的测试) 98, 201

\pdfbookmark[2]{K}{K}\medskip\textbf{\large{K}}

\hangindent=2pc \hangafter=1 Kernighan, B. W. 66, 130, 152, 186

\hangindent=2pc \hangafter=1 Kernighan, M. D. vi 

\hangindent=2pc \hangafter=1 Knuth, Donald Ervin 82, 179, 198

\hangindent=2pc \hangafter=1 KWIC index (KWIC 索引) 122

\hangindent=2pc \hangafter=1 \verb'kwic' program (程序 \verb'kwic') 123, 203

\pdfbookmark[2]{L}{L}\medskip\textbf{\large{L}}

\hangindent=2pc \hangafter=1 language, asembly (汇编语言) 133

\hangindent=2pc \hangafter=1 language features, new (新的语言特性) v, 79, 182

\hangindent=2pc \hangafter=1 language, Forth (Forth 语言) 142

\hangindent=2pc \hangafter=1 language, \verb'grap' (\verb'grap' 语言) 139, 152

\hangindent=2pc \hangafter=1 language, \verb'graph' (\verb'graph' 语言) 135

\hangindent=2pc \hangafter=1 language, ICON (ICON 语言) 186

\hangindent=2pc \hangafter=1 language, pattern-directed (面向模式的语言) 138, 140, 152, 156, 181

\hangindent=2pc \hangafter=1 language, \verb'pic' (\verb'pic' 语言) 139

\hangindent=2pc \hangafter=1 language, Postscript (Postscript 语言) 142

\hangindent=2pc \hangafter=1 language processor model (语言处理器模型) 131

\hangindent=2pc \hangafter=1 language, \textit{q} query (\textit{q}
查询语言) 102, 107

\hangindent=2pc \hangafter=1 language, query (查询语言) 99

\hangindent=2pc \hangafter=1 language, REXX (REXX 语言) 186

\hangindent=2pc \hangafter=1 language, SNOBOL4 (SNOBOL4 语言) 50, 182, 186

\hangindent=2pc \hangafter=1 language, \verb'sortgen' (\verb'sortgen' 语言) 140

\hangindent=2pc \hangafter=1 leap year computation (闰年的计算) 194

\hangindent=2pc \hangafter=1 leftmost longest match (最左最长匹配) 42, 60, 80

% \hangindent=2pc \hangafter=1 \marginpar{208}
\hangindent=2pc \hangafter=1 \verb'length' function (函数 \verb'length') 13-14

\hangindent=2pc \hangafter=1 letters, form (格式信函) 100

\hangindent=2pc \hangafter=1 \verb'lex' lexical analyzer generator
(\verb'lex' 词法分析器生成程序) 152, 181, 186

\hangindent=2pc \hangafter=1 lexical analysis (词法分析) 131, 133

\hangindent=2pc \hangafter=1 limits, implementation (实现的限制) 59, 61-62, 191

\hangindent=2pc \hangafter=1 Linderman. J. P. vi

\hangindent=2pc \hangafter=1 linear algorithm (线性算法) 157, 183

\hangindent=2pc \hangafter=1 linear order (线序) 170

\hangindent=2pc \hangafter=1 lines versus records (行与记录) 21, 60

\hangindent=2pc \hangafter=1 little languages (小语言) iv-v, 131, 152, 156, 159

\hangindent=2pc \hangafter=1 local variables (局部变量) 54, 116, 182

\hangindent=2pc \hangafter=1 \verb'log' function (函数 \verb'log') 39

\hangindent=2pc \hangafter=1 logarithm, computation of base-10 (计算以 10
为底的对数) 39

\hangindent=2pc \hangafter=1 logical expression, value of (逻辑表达式的值) 37

\hangindent=2pc \hangafter=1 logical operators (逻辑运算符) 10, 31, 37

\hangindent=2pc \hangafter=1 logical operators, precedence of (逻辑运算符
的优先级) 32

\hangindent=2pc \hangafter=1 long statements, continuing (长语句的延续)
15, 22, 188

\hangindent=2pc \hangafter=1 \verb'lorder' command (命令 \verb'lorder') 170

\hangindent=2pc \hangafter=1 \verb'ls' command (命令 \verb'ls') 177

\hangindent=2pc \hangafter=1 Lukasiewicz. J. 142

\pdfbookmark[2]{M}{M}\medskip\textbf{\large{M}}

\hangindent=2pc \hangafter=1 machine dependency (机器依赖) 35-36, 44-45, 51, 183

\hangindent=2pc \hangafter=1 \verb'make' command (命令 \verb'make') 175

\hangindent=2pc \hangafter=1 \verb'make' program (程序 \verb'make') 178

\hangindent=2pc \hangafter=1 \verb'make' update algorithm (更新程序
\verb'make') 176

\hangindent=2pc \hangafter=1 \verb'makefile' dependency description
(\verb'makefile' 的依赖关系描述) 175

\hangindent=2pc \hangafter=1 \verb'makeprof' program (程序 \verb'makeprof') 167

\hangindent=2pc \hangafter=1 manuscripts, cross-references in
(手稿的交叉引用) 120

\hangindent=2pc \hangafter=1 Martin, R. L. vi

\hangindent=2pc \hangafter=1 \verb'match' function (函数 \verb'match')
35,41,149,182, 189, 196

\hangindent=2pc \hangafter=1 match, leftmost longest (最左最长匹配) 42, 60, 80

\hangindent=2pc \hangafter=1 match operator, \verb'~' (匹配运算符 \verb'~')
25, 27, 31, 37, 40

\hangindent=2pc \hangafter=1 \verb'max' function (函数 \verb'max') 53

\hangindent=2pc \hangafter=1 Mcllroy. M. D. vi, 130

\hangindent=2pc \hangafter=1 metacharacters, regular expression
(正则表达式的元字符) 28, 191

\hangindent=2pc \hangafter=1 Miller. W. 179

\hangindent=2pc \hangafter=1 model, language processor (语言处理器模型) 131

\hangindent=2pc \hangafter=1 Moscovitz. H. S. vi

\hangindent=2pc \hangafter=1 MS-DOS vi, 26

\hangindent=2pc \hangafter=1 multidimensional arrays (多维数组)
52, 108, 114, 116, 182

\hangindent=2pc \hangafter=1 multifile database (多文件数据库) 102

\hangindent=2pc \hangafter=1 multiline records (多行记录) iv, 60-61, 82

\hangindent=2pc \hangafter=1 multiple assignment (多重赋值) 39

\hangindent=2pc \hangafter=1 multiple \verb'BEGIN' and \verb'END' (多重\verb'BEGIN'
与 \verb'END') 23, 169

\hangindent=2pc \hangafter=1 Myers. E. 179

\pdfbookmark[2]{N}{N}\medskip\textbf{\large{N}}

\hangindent=2pc \hangafter=1 $n \log n$ algorithm ($n \log n$ 算法) 162, 165

\hangindent=2pc \hangafter=1 named fields (命名字段) 102, 107

\hangindent=2pc \hangafter=1 names, rules for variable (变量的取名规则) 35

\hangindent=2pc \hangafter=1 name-value data (\mbox{名字}-值 类型的数据) 86

\hangindent=2pc \hangafter=1 natural join (自然连接) 103

\hangindent=2pc \hangafter=1 new language features (新语言的特性) v, 79, 182

\hangindent=2pc \hangafter=1 newline as field separator (换行符作为字段分隔符)
61,  83-84

\hangindent=2pc \hangafter=1 newline character, \verb'\n' (换行符) 8,31,79

\hangindent=2pc \hangafter=1 \verb'next' statement (\verb'next' 语句) 49

\hangindent=2pc \hangafter=1 \verb'NF', side-effects on (\verb'NF'
的副作用) 36, 61

\hangindent=2pc \hangafter=1 \verb'NF' variable (变量 \verb'NF') 6, 14, 35-36, 61

\hangindent=2pc \hangafter=1 \verb'nm' command (命令 \verb'nm') 73

\hangindent=2pc \hangafter=1 \verb'nm.format' program (程序
\verb'nm.format') 74


\hangindent=2pc \hangafter=1 nonexistent field (不存在的字段) 36, 45, 192

\hangindent=2pc \hangafter=1 nonmatch operator, \verb'!~' (不匹配运算符
\verb'!~') 25, 27, 31, 37, 40

\hangindent=2pc \hangafter=1 nonterminal symbol (非终结符) 113.145

\hangindent=2pc \hangafter=1 NOT operator, \verb'!' (逻辑 ``非'' 运算符)
10,31,37

\hangindent=2pc \hangafter=1 notation, exponential (指数记号) 35

\hangindent=2pc \hangafter=1 notation, infix (中缀表示法) 142, 145

\hangindent=2pc \hangafter=1 notation, reverse-Polish (逆波兰式表示法) 142

\hangindent=2pc \hangafter=1 \verb'NR' variable (变量 \verb'NR')
6, 12, 14, 35-36, 61

\hangindent=2pc \hangafter=1 null string (空字符串) 13,24,42, 114, 192

\hangindent=2pc \hangafter=1 number, coercion to (强制转换成数值) 45

\hangindent=2pc \hangafter=1 number or string (数值或字符串) 44

\hangindent=2pc \hangafter=1 number, regular expression for floating-point
(浮点数的正则表达式) 30,40

\hangindent=2pc \hangafter=1 number to string coercion
(数值强制转换成字符串) 25

\hangindent=2pc \hangafter=1 number to string conversion (数值到字符串的转换)
35,  44, 192

\hangindent=2pc \hangafter=1 numbers to words, convert (数值转换成单词) 76

\hangindent=2pc \hangafter=1 numeric comparison (数值比较) 25-26, 44

\hangindent=2pc \hangafter=1 numeric constant (数值常量) 35

\hangindent=2pc \hangafter=1 numeric subscripts (数值下标) 52

\hangindent=2pc \hangafter=1 numeric value of a string (字符串的数值值) 45

\hangindent=2pc \hangafter=1 numeric variables (数字形式的值) 44

\hangindent=2pc \hangafter=1 \verb'numtowords' function (函数
\verb'numtowords') 76, 194

\pdfbookmark[2]{O}{O}\medskip\textbf{\large{O}}

\hangindent=2pc \hangafter=1 \verb'OFMT' variable (变量 \verb'OFMT') 36,45

\hangindent=2pc \hangafter=1 \verb'OFS' variable (变量 \verb'OFS')
35-36, 43, 55-56

\hangindent=2pc \hangafter=1 one-liners (单行程序) 17, 181

\hangindent=2pc \hangafter=1 operator, \verb'&&' AND (逻辑 ``与'' 运算符)
10,31,37, 158

\hangindent=2pc \hangafter=1 operator, \verb'%=' assignment (赋值运算符) 38

\hangindent=2pc \hangafter=1 operator, \verb'*=' assignment (赋值运算符) 38

\hangindent=2pc \hangafter=1 operator, \verb'+=' assignment (赋值运算符) 38

\hangindent=2pc \hangafter=1 operator, \verb'-=' assignment (赋值运算符) 38

\hangindent=2pc \hangafter=1 operator, \verb'/=' assignment (赋值运算符) 38

\hangindent=2pc \hangafter=1 operator, \verb'^=' assignment (赋值运算符) 38

\hangindent=2pc \hangafter=1 operator, \verb'=' assignment (赋值运算符) 38, 44

\hangindent=2pc \hangafter=1 operator, \verb'>' comparison (比较) 9

\hangindent=2pc \hangafter=1 operator, \verb'>=' comparison (比较) 9

\hangindent=2pc \hangafter=1 operator, \verb'==' comparison (比较) 9, 44

\hangindent=2pc \hangafter=1 operator, \verb'--' decrement (自减运算符)
39, 70, 112

\hangindent=2pc \hangafter=1 operator, \verb'^' exponentiation (指数运算符)
15, 36, 46

\hangindent=2pc \hangafter=1 operator, \verb'++' increment (自增运算符)
39, 146, 158

\hangindent=2pc \hangafter=1 operator, \verb'~' match (匹配运算符) 25, 27, 31, 37, 40

\hangindent=2pc \hangafter=1 operator, \verb'!~' nonmatch (不匹配运算符)
25, 27, 31,37,40

\hangindent=2pc \hangafter=1 operator, \verb'~' NOT (逻辑 ``非'' 运算符)
10, 31. 37

\hangindent=2pc \hangafter=1 operator, \verb'||' OR (逻辑 ``或'' 运算符)
10,31,37

\hangindent=2pc \hangafter=1 operator, \verb'%' remainder (取余) 36, 46

\hangindent=2pc \hangafter=1 operator, concatenation (拼接) 40, 43,  182

\hangindent=2pc \hangafter=1 operator, \verb'in' (成员运算符 \verb'in') 52, 192

\hangindent=2pc \hangafter=1 operators, arithmetic (算术运算符) 36, 44

\hangindent=2pc \hangafter=1 operators, assignment (赋值运算符) 38

\hangindent=2pc \hangafter=1 operators, associativity of (运算符的结合性) 46

\hangindent=2pc \hangafter=1 operators, comparison (比较运算符) 36

\hangindent=2pc \hangafter=1 operators, logical (逻辑运算符) 10, 31, 37

\hangindent=2pc \hangafter=1 operators, precedence of (运算符的优先级) 46

\hangindent=2pc \hangafter=1 operators, precedence of regular expression
(正则表达式的优先级) 30

\hangindent=2pc \hangafter=1 operators, relational (关系运算符) 25, 37

\hangindent=2pc \hangafter=1 operators, summary of (运算符总结) 190

\hangindent=2pc \hangafter=1 operators, table of arithmetic (算术运算符汇总,
表格) 46

\hangindent=2pc \hangafter=1 operators, table of comparison (比较运算符
汇总, 表格) 25

\hangindent=2pc \hangafter=1 option, \verb'-f' (选项 \verb'-f') 4-5,63,65, 187

\hangindent=2pc \hangafter=1 option, \verb'-F' (选项 \verb'-F') 60,63, 187

\hangindent=2pc \hangafter=1 OR operator, \verb'||' (逻辑 ``或'' 运算符)
10, 31, 37

\hangindent=2pc \hangafter=1 \verb'ORS' variable (变量 \verb'ORS') 36, 55-56, 83

\hangindent=2pc \hangafter=1 output field separator (输出字段分隔符)
6, 35, 39, 55-56

\hangindent=2pc \hangafter=1 output file, standard (标准输出) 5, 56

\hangindent=2pc \hangafter=1 output into pipes (输送给管道) 8, 58

\hangindent=2pc \hangafter=1 output record separator (输出记录分隔符)
6, 55-56,  83

\hangindent=2pc \hangafter=1 output redirection, \verb'>' (输出重定向)
56, 58, 188

\hangindent=2pc \hangafter=1 output redirection, \verb'>>' (输出重定向) 56, 58, 188

\hangindent=2pc \hangafter=1 output redirection, \verb'|'
(输出数据的字段分隔符) 58, 188

\hangindent=2pc \hangafter=1 output statements, summary of (输出语句总结) 55

\pdfbookmark[2]{P}{P}\medskip\textbf{\large{P}}

\hangindent=2pc \hangafter=1 \verb'p12check' (程序 \verb'p12check') program 77

\hangindent=2pc \hangafter=1 parameter, array (把数组作为参数) 54

\hangindent=2pc \hangafter=1 parameter list (参数列表) 53, 116

\hangindent=2pc \hangafter=1 parameter, scalar (标题参数) 54

\hangindent=2pc \hangafter=1 parameters, formal (形式参数) 54

\hangindent=2pc \hangafter=1 parenthesis-free notation (不需要括号的表示法) 142

\hangindent=2pc \hangafter=1 Parnas, D. L. 123, 130

\hangindent=2pc \hangafter=1 parser generator, \verb'yacc'
(语法分析器生成程序 \verb'yacc') 152, 175

\hangindent=2pc \hangafter=1 parsing, recursive-descent (递归下降语法分析)
145, 147-148

\hangindent=2pc \hangafter=1 partial order (偏序) 170

\hangindent=2pc \hangafter=1 partitioning step, quicksort
(快速排序的划分操作) 160

\hangindent=2pc \hangafter=1 \verb'passwd' program (程序 \verb'passwd') 78

\hangindent=2pc \hangafter=1 password file, check (密码文件检查) 78

\hangindent=2pc \hangafter=1 pattern, \verb'BEGIN' (模式 \verb'BEGIN') 11,23,63

\hangindent=2pc \hangafter=1 pattern, \verb'END' (模式 \verb'END') 11,23,49

\hangindent=2pc \hangafter=1 pattern, range (范围模式) 32, 85, 187

\hangindent=2pc \hangafter=1 pattern, string-matching (字符串匹配模式) 26

\hangindent=2pc \hangafter=1 pattern-action cycle (\patact 循环) 3, 21

\hangindent=2pc \hangafter=1 pattern-action statement (\patact 语句)
iii, 2, 21, 34, 53, 187

\hangindent=2pc \hangafter=1 pattern-directed language (面向模式的语言)
138,  140, 152, 156, 181

\hangindent=2pc \hangafter=1 patterns, compound (复合模式) 31

\hangindent=2pc \hangafter=1 patterns, summary of (模式总结) 23, 187

\hangindent=2pc \hangafter=1 patterns, summary of string-matching
(字符串匹配模式的总结) 27

\hangindent=2pc \hangafter=1 patterns, table of (模式汇总, 表格) 33

\hangindent=2pc \hangafter=1 \verb'percent' program (程序 \verb'percent') 70

\hangindent=2pc \hangafter=1 performance, heapsort (堆排序的性能) 165

\hangindent=2pc \hangafter=1 performance, insertion sort (插入排序的性能) 158

\hangindent=2pc \hangafter=1 performance measurements, table of (性能测试
汇总, 表格) 183

\hangindent=2pc \hangafter=1 performance, quicksort (快速排序的性能) 162

\hangindent=2pc \hangafter=1 \verb'permute' function (函数 \verb'permute') 199

\hangindent=2pc \hangafter=1 \verb'pic' language (\verb'pic' 语言) 139

\hangindent=2pc \hangafter=1 Pike, R. 66

\hangindent=2pc \hangafter=1 pipe, \verb'|' input (输入管道) 62, 76

\hangindent=2pc \hangafter=1 pipeline, indexing (索引的流水线) 129

\hangindent=2pc \hangafter=1 Pipes, output into (输出给管道) 8, 58

\hangindent=2pc \hangafter=1 Poage, J. 186

\hangindent=2pc \hangafter=1 Polish notation (波兰表示法) 142

\hangindent=2pc \hangafter=1 Polonsky, I. 186

\hangindent=2pc \hangafter=1 Postscript language (Postscript 语言) 142

\hangindent=2pc \hangafter=1 \verb'pr' command (命令 \verb'pr') 175

\hangindent=2pc \hangafter=1 \verb'prchecks' program (程序 \verb'prchecks') 75

\hangindent=2pc \hangafter=1 precedence of logical operators
(逻辑运算符的优先级) 32

\hangindent=2pc \hangafter=1 precedence of operators (运算符的优先级) 46

\hangindent=2pc \hangafter=1 precedence of regular expression operators
(正则表达式运算符的优先级) 30

\hangindent=2pc \hangafter=1 precision, floating-point (浮点数精度) 35, 191

\hangindent=2pc \hangafter=1 predecessor node (前驱结点) 170

\hangindent=2pc \hangafter=1 \verb'prefix' function (函数 \verb'prefix') 105

\hangindent=2pc \hangafter=1 \verb'prep1' program (程序 \verb'prep1') 90

\hangindent=2pc \hangafter=1 \verb'prep2' program (程序 \verb'prep2') 91

\hangindent=2pc \hangafter=1 \verb'prep3' program (程序 \verb'prep3') 93

\hangindent=2pc \hangafter=1 primary expressions (初等表达式) 34

\hangindent=2pc \hangafter=1 \verb'print >'\textit{file} 90, 188

\hangindent=2pc \hangafter=1 \verb'print' statement (\verb'print' 语句) 5, 55

\hangindent=2pc \hangafter=1 \verb'printf', \verb'%%' in 79

\hangindent=2pc \hangafter=1 \verb'printf' examples, table of
(\verb'printf' 的例子) 57

\hangindent=2pc \hangafter=1 \verb'printf' specifications, summary of
(\verb'printf' 的规范说明总结) 189

% \hangindent=2pc \hangafter=1 \marginpar{209}

\hangindent=2pc \hangafter=1 \verb'printf' specifications, table of
(\verb'printf' 的规范说明总结, 表格) 57


\hangindent=2pc \hangafter=1 \verb'printf' statement (\verb'printf' 语句)
7, 24, 56, 98

\hangindent=2pc \hangafter=1 printing a blank line (打印一个空行) 11, 55

\hangindent=2pc \hangafter=1 printing checks (打印支票) 74

\hangindent=2pc \hangafter=1 printing error messages (打印错误信息) 59

\hangindent=2pc \hangafter=1 \verb'printprof' program (程序
\verb'printprof') 168

\hangindent=2pc \hangafter=1 priority queue (优先级队列) 162

\hangindent=2pc \hangafter=1 processor, \verb'#lnclude' (\verb'#include'
的处理器) 62, 64

\hangindent=2pc \hangafter=1 profiling (剖析) 167

\hangindent=2pc \hangafter=1 profiling heapsort (堆排序的剖析) 168-169

\hangindent=2pc \hangafter=1 program, \verb'addcomma' (程序 \verb'addcomma')
72, 194

\hangindent=2pc \hangafter=1 program, \verb'arith' (程序 \verb'arith') 117

\hangindent=2pc \hangafter=1 program, \verb'asm' (程序 \verb'asm') 134, 203

\hangindent=2pc \hangafter=1 program, \verb'awk.parser' 
(程序 \verb'awk.parser') 149

\hangindent=2pc \hangafter=1 program, batch sort test
(批处理的排序测试程序) 155

\hangindent=2pc \hangafter=1 program, \verb'bridge' (程序 \verb'bridge') 199

\hangindent=2pc \hangafter=1 program, \verb'bundle'  (程序 \verb'bundle') 81

\hangindent=2pc \hangafter=1 program, \verb'calc1' (程序 \verb'calc1') 143

\hangindent=2pc \hangafter=1 program, \verb'calc2' (程序 \verb'calc2') 144

\hangindent=2pc \hangafter=1 program, \verb'calc3' (程序 \verb'calc3') 146

\hangindent=2pc \hangafter=1 program, \verb'check1' (程序 \verb'check1') 87

\hangindent=2pc \hangafter=1 program, \verb'check2' (程序 \verb'check2') 87

\hangindent=2pc \hangafter=1 program, \verb'check3' (程序 \verb'check3') 88

\hangindent=2pc \hangafter=1 program, \verb'checkgen' (程序 \verb'checkgen') 79

\hangindent=2pc \hangafter=1 program, \verb'cliche' (程序 \verb'cliche') 113

\hangindent=2pc \hangafter=1 program, \verb'colcheck' (程序 \verb'colcheck') 77

\hangindent=2pc \hangafter=1 program, \verb'compat' (程序 \verb'compat') 80

\hangindent=2pc \hangafter=1 program, \verb'echo' (程序 \verb'echo') 63

\hangindent=2pc \hangafter=1 program, \verb'field' (程序 \verb'field') 66

\hangindent=2pc \hangafter=1 program, \verb'fmt' (程序 \verb'fmt') 120

\hangindent=2pc \hangafter=1 program, \verb'fmt.just' (程序 \verb'fmt.just')
202

\hangindent=2pc \hangafter=1 program, \verb'form1' (程序 \verb'form1') 91

\hangindent=2pc \hangafter=1 program, \verb'form2' (程序 \verb'form2') 92

\hangindent=2pc \hangafter=1 program, \verb'form3' (程序 \verb'form3') 94

\hangindent=2pc \hangafter=1 program, \verb'form4' (程序 \verb'form4') 96

\hangindent=2pc \hangafter=1 program \verb'format' (程序 \verb'format')
11, 22, 34, 47, 53, 188

\hangindent=2pc \hangafter=1 program, \verb'form.gen' (程序 \verb'form.gen') 101

\hangindent=2pc \hangafter=1 program \verb'generation' (程序 \verb'generation')
v, 79, 121, 167

\hangindent=2pc \hangafter=1 program, \verb'graph' (程序 \verb'graph') 137

\hangindent=2pc \hangafter=1 program, \verb'histogram' (程序 \verb'histogram')
70, 193

\hangindent=2pc \hangafter=1 program, \verb'info' (程序 \verb'info') 100

\hangindent=2pc \hangafter=1 program, \verb'interest' (程序 \verb'interest') 15

\hangindent=2pc \hangafter=1 program, \verb'ix.collapse' (程序
\verb'ix.collapse') 126

\hangindent=2pc \hangafter=1 program, \verb'ix.format' (程序 \verb'ix.format')
129

\hangindent=2pc \hangafter=1 program, \verb'ix.genkey' (程序 \verb'ix.genkey')
128

\hangindent=2pc \hangafter=1 program, \verb'ix.rotate' (程序 \verb'ix.rotate')
127

\hangindent=2pc \hangafter=1 program, \verb'ix.sort1' (程序 \verb'ix.sort1') 126

\hangindent=2pc \hangafter=1 program, \verb'ix.sort2' (程序 \verb'ix.sort2') 128

\hangindent=2pc \hangafter=1 program, \verb'join' (程序 \verb'join') 104

\hangindent=2pc \hangafter=1 program, \verb'kwic' (程序 \verb'kwic') 123. 203

\hangindent=2pc \hangafter=1 program, \verb'make' (程序 \verb'make') 178

\hangindent=2pc \hangafter=1 program, \verb'makeprof' (程序 \verb'makeprof') 167

\hangindent=2pc \hangafter=1 program, \verb'nm.format' (程序 \verb'nm.format')
74

\hangindent=2pc \hangafter=1 program, \verb'p12check' (程序 \verb'p12check') 77

\hangindent=2pc \hangafter=1 program, \verb'passwd' (程序 \verb'passwd') 78

\hangindent=2pc \hangafter=1 program, \verb'percent' (程序 \verb'percent') 70

\hangindent=2pc \hangafter=1 program, \verb'prchecks' (程序 \verb'prchecks') 75

\hangindent=2pc \hangafter=1 program, \verb'prep1' (程序 \verb'prep1') 90

\hangindent=2pc \hangafter=1 program, \verb'prep2' (程序 \verb'prep2') 91

\hangindent=2pc \hangafter=1 program, \verb'prep3' (程序 \verb'prep3') 93

\hangindent=2pc \hangafter=1 program, \verb'printprof' (程序
\verb'printprof') 168

\hangindent=2pc \hangafter=1 program, \verb'qawk' (程序 \verb'qawk') 109

\hangindent=2pc \hangafter=1 program, \verb'quiz'  (程序 \verb'quiz') 118

\hangindent=2pc \hangafter=1 program, \verb'rtsort'  (程序 \verb'rtsort') 174

\hangindent=2pc \hangafter=1 program, \verb'sentgen'  (程序 \verb'sentgen')
115,200-201

\hangindent=2pc \hangafter=1 program, \verb'seq'  (程序 \verb'seq') 64

\hangindent=2pc \hangafter=1 program, \verb'sortgen'  (程序 \verb'sortgen') 141

\hangindent=2pc \hangafter=1 program, \verb'sum1'  (程序 \verb'sum1') 68

\hangindent=2pc \hangafter=1 program, \verb'sum2'  (程序 \verb'sum2') 68

\hangindent=2pc \hangafter=1 program, \verb'sum3'  (程序 \verb'sum3') 69

\hangindent=2pc \hangafter=1 program, \verb'suncomma' (程序 \verb'suncomma') 71

\hangindent=2pc \hangafter=1 program, \verb'table' (程序 \verb'table') 98

\hangindent=2pc \hangafter=1 program, \verb'table1' (程序 \verb'table1') 196

\hangindent=2pc \hangafter=1 program, test framework (测试框架) 159

\hangindent=2pc \hangafter=1 program, \verb'transpose' (程序
\verb'transpose') 204

\hangindent=2pc \hangafter=1 program, \verb'tsort' (程序 \verb'tsort') 172

\hangindent=2pc \hangafter=1 program, \verb'unbundle' (程序
\verb'unbundle') 82

\hangindent=2pc \hangafter=1 program, word count (单词计数程序) 14, 119

\hangindent=2pc \hangafter=1 program, \verb'wordfreq' (程序
\verb'wordfreq') 119

\hangindent=2pc \hangafter=1 program, \verb'xref' (程序 \verb'xref') 122

\hangindent=2pc \hangafter=1 prompt character (提示字符) 2

\hangindent=2pc \hangafter=1 prototyping (原型) iii, v, 78, 152, 185

\hangindent=2pc \hangafter=1 pseudo-code (伪代码) iv, 153

\hangindent=2pc \hangafter=1 \verb'ptx' command (命令 \verb'ptx') 123

\hangindent=2pc \hangafter=1 pushback, input (输入的撤回) 105, 110

\pdfbookmark[2]{Q}{Q}\medskip\textbf{\large{Q}}

\hangindent=2pc \hangafter=1 \textit{q} query language (\textit{q}
查询语言) 102, 107

\hangindent=2pc \hangafter=1 \verb'qawk' program (程序 \verb'qawk') 109

\hangindent=2pc \hangafter=1 \verb'qawk' query processor (\verb'qawk'
的查询处理程序) 108

\hangindent=2pc \hangafter=1 \verb'qsort' function (函数 \verb'qsort') 161

\hangindent=2pc \hangafter=1 quadratic algorithm (平方程序) 157,162, 183

\hangindent=2pc \hangafter=1 query language (查询语言) 99

\hangindent=2pc \hangafter=1 queue (队列) 171

\hangindent=2pc \hangafter=1 queue, priority (优先级队列) 162

\hangindent=2pc \hangafter=1 quicksort algorithm (快速排序算法) 160

\hangindent=2pc \hangafter=1 quicksort partitioning step
(快速排序的划分操作) 160

\hangindent=2pc \hangafter=1 quicksort performance (快速排序的性能) 162

\hangindent=2pc \hangafter=1 \verb'quiz' program (程序 \verb'quiz') 118

\hangindent=2pc \hangafter=1 quotes, \verb"' '" (单引号) 2, 4-5, 65, 100

\hangindent=2pc \hangafter=1 quoting in regular expressions
(正则表达式中的引用) 29-30, 41, 43

\pdfbookmark[2]{R}{R}\medskip\textbf{\large{R}}

\hangindent=2pc \hangafter=1 \verb'rand' function (函数 \verb'rand') 40, 111

\hangindent=2pc \hangafter=1 \verb'rand', initializing (\verb'rand'
的初始化) 111

\hangindent=2pc \hangafter=1 \verb'randint' function (函数 \verb'randint') 111

\hangindent=2pc \hangafter=1 \verb'randlet' function (函数 \verb'randlet') 112

\hangindent=2pc \hangafter=1 random sentences (随机句子) 113

\hangindent=2pc \hangafter=1 range pattern (范围模式) 32, 85, 187

\hangindent=2pc \hangafter=1 record separator, output (输出的记录分隔符)
6, 55-56, 83

\hangindent=2pc \hangafter=1 record variable, \verb'$0' (记录分隔符,
\verb'$0') 5, 35

\hangindent=2pc \hangafter=1 records, lines versus (记录与行) 21,60

\hangindent=2pc \hangafter=1 records, multiline (多行记录) iv, 60-61,82

\hangindent=2pc \hangafter=1 records with headers (带有头部的记录) 85

\hangindent=2pc \hangafter=1 recursion elimination (消除递归) 200, 204

\hangindent=2pc \hangafter=1 recursive function (递归函数) 54, 71, 76, 115, 160

\hangindent=2pc \hangafter=1 recursive-descent parsing (递归下降语法分析) 145, 147-148

\hangindent=2pc \hangafter=1 redirection, \verb'>' output (输出重定向) 56, 58, 188

\hangindent=2pc \hangafter=1 redirection, \verb'>>' output (输出重定向) 56, 58, 188

\hangindent=2pc \hangafter=1 redirection, \verb'|' output (输出重定向) 58, 188

\hangindent=2pc \hangafter=1 regular expression, \verb'.'
(正则表达式中的句点) 28

\hangindent=2pc \hangafter=1 regular expression, \verb'$'
(正则表达式中的美元符) 28, 119

\hangindent=2pc \hangafter=1 regular expression, \verb'^' (正则表达式中的
\verb'^') 28, 119

\hangindent=2pc \hangafter=1 regular expression, \verb'()' (正则表达式中的
\verb'()') 29

\hangindent=2pc \hangafter=1 regular expression, \verb'['...\verb']'
(正则表达式中的 \verb'['...\verb']') 29

\hangindent=2pc \hangafter=1 regular expression, \verb'[^'...\verb']'
(正则表达式中的 \verb'[^'...\verb']') 29

\hangindent=2pc \hangafter=1 regular expression, \verb'|' (正则表达式中的
\verb'|') 29, 32

\hangindent=2pc \hangafter=1 regular expression, \verb'*' (正则表达式中的
\verb'*') 30

\hangindent=2pc \hangafter=1 regular expression, \verb'+' (正则表达式中的
\verb'+') 30

\hangindent=2pc \hangafter=1 regular expression, \verb'?' (正则表达式中的
\verb'?') 30

\hangindent=2pc \hangafter=1 regular expression character class (正则表达式中 
的字符类) 29

\hangindent=2pc \hangafter=1 regular expression, concatenation in
(正则表达式中的拼接) 29

\hangindent=2pc \hangafter=1 regular expression, dynamic (动态的正则表达式) 40, 101, 184

\hangindent=2pc \hangafter=1 regular expression examples (正则表达式的例子) 30

\hangindent=2pc \hangafter=1 regular expression field separator
(正则表达式字段分隔符) 52, 60, 80, 135

\hangindent=2pc \hangafter=1 regular expression for floatingpoint number
(浮点数的正则表达式) 30, 40

\hangindent=2pc \hangafter=1 regular expression metacharacters (正则表达式的元字符) 28, 191

\hangindent=2pc \hangafter=1 regular expression operators, precedence of
(正则表达式运算符的优先级) 30

\hangindent=2pc \hangafter=1 regular expressions in data
(数据中的正则表达式) 118

\hangindent=2pc \hangafter=1 regular expressions, quoting in (正则表达式中的 
引用) 29-30,41,43

\hangindent=2pc \hangafter=1 regular expressions, strings as
(字符串作为正则表达式) 40

\hangindent=2pc \hangafter=1 regular expressions, summary of
(正则表达式的总结) 28, 191

\hangindent=2pc \hangafter=1 regular expressions, table of (正则表达式汇总,
表格) 32

\hangindent=2pc \hangafter=1 relation, universal (全局关系表) 107

\hangindent=2pc \hangafter=1 relational databases (关系数据库) iv, 102

\hangindent=2pc \hangafter=1 relational operators (关系运算符) 25, 37

\hangindent=2pc \hangafter=1 \verb'relfile' database description
(数据库描述文件 \verb'relfile') 106

\hangindent=2pc \hangafter=1 remainder operator, \verb'%' (取余运算符) 36,46

\hangindent=2pc \hangafter=1 report generation (生成报表) 89

\hangindent=2pc \hangafter=1 \verb'return' statement (\verb'return' 语句) 53

\hangindent=2pc \hangafter=1 reverse input line order (逆转输入行的顺序) 50

\hangindent=2pc \hangafter=1 \verb'reverse' program (程序 \verb'reverse') 16-17

\hangindent=2pc \hangafter=1 reverse-Polish notation (逆波兰式表示法) 142

\hangindent=2pc \hangafter=1 REXX language (REXX 语言) 186

\hangindent=2pc \hangafter=1 Ritchie, D. M. 66, 186

\hangindent=2pc \hangafter=1 \verb'RLENGTH' variable (变量 \verb'RLENGTH')
35-36,41

\hangindent=2pc \hangafter=1 rounding to nearest integer (取整) 40

\hangindent=2pc \hangafter=1 \verb'RS' variable (变量 \verb'RS') 36, 60, 83-84

\hangindent=2pc \hangafter=1 \verb'RSTART' variable (变量 \verb'RSTART')
35-36,41,189

\hangindent=2pc \hangafter=1 \verb'rtsort' program (程序 \verb'rtsort') 174

\hangindent=2pc \hangafter=1 rules for variable names (变量的取名规则) 35

\hangindent=2pc \hangafter=1 running an awk program (运行 awk 程序) 3

\hangindent=2pc \hangafter=1 running time of awk programs (awk
程序的运行时间) 183

\pdfbookmark[2]{S}{S}\medskip\textbf{\large{S}}

\hangindent=2pc \hangafter=1 scaffolding (脚手架) 153, 156, 179

\hangindent=2pc \hangafter=1 scalar parameter (标量参数) 54

\hangindent=2pc \hangafter=1 Schmitt, G. vi

\hangindent=2pc \hangafter=1 scientific notation (科学记数法) 35

\hangindent=2pc \hangafter=1 Scribe formatter (排版程序 Scribe) 124

\hangindent=2pc \hangafter=1 \verb'sed' command (命令 \verb'sed')
v, 181, 184, 186

\hangindent=2pc \hangafter=1 self-identifying data (自描述数据) 86

\hangindent=2pc \hangafter=1 semicolon (分号) 11, 22, 34, 47, 53, 187

\hangindent=2pc \hangafter=1 semicolon as empty statement (空语句) 50

\hangindent=2pc \hangafter=1 semicolon before \verb'else' (\verb'else'
前的分号) 47

\hangindent=2pc \hangafter=1 sentence generation (语句的生成) 114

\hangindent=2pc \hangafter=1 sentences, random (随机句子) 113

\hangindent=2pc \hangafter=1 \verb'sentgen' program (程序 \verb'sentgen')
115, 200-201

\hangindent=2pc \hangafter=1 separator, blank line (分隔符, 空行) 83

\hangindent=2pc \hangafter=1 separator, default field (默认的字段分隔符) 5, 24

\hangindent=2pc \hangafter=1 separator, input field (输入的字段分隔符)
24, 35, 39, 60

\hangindent=2pc \hangafter=1 separator, output field (输出的字段隔符)
6, 35, 39, 55-56

\hangindent=2pc \hangafter=1 separator, output record (输出的记录分隔符)
6, 55-56,  83

\hangindent=2pc \hangafter=1 \verb'seq' program (程序 \verb'seq') 64

\hangindent=2pc \hangafter=1 Sethi, R. 130, 152, 186

\hangindent=2pc \hangafter=1 shell command interpreter (shell 解释器) 4, 65, 99

\hangindent=2pc \hangafter=1 side-effects of assignment (赋值的副作用) 43

\hangindent=2pc \hangafter=1 side-effects of \verb'getline' (\verb'getline'
的副作用) 61

\hangindent=2pc \hangafter=1 side-effects of input (输入的副作用) 35

\hangindent=2pc \hangafter=1 side-effects of \verb'sub' (\verb'sub'
的副作用) 43

\hangindent=2pc \hangafter=1 side-effects of test (条件判断的副作用) 52,192

\hangindent=2pc \hangafter=1 side-effects on \verb'$0' (\verb'$0' 的副作用)
36, 43

\hangindent=2pc \hangafter=1 side-effects on \verb'NF' (\verb'NF' 的副作用)
36, 61

\hangindent=2pc \hangafter=1 \verb'sin' function (函数 \verb'sin') 39

\hangindent=2pc \hangafter=1 SNOBOL4 language (SNOBOL4 语言) 50, 182, 186

\hangindent=2pc \hangafter=1 \verb'sort' command (命令 \verb'sort')
8, 58, 84, 90

\hangindent=2pc \hangafter=1 sort key (排序键) 91, 127, 140

\hangindent=2pc \hangafter=1 \verb'sort' options (\verb'sort' 的选项)
90, 94, 124, 126, 140

\hangindent=2pc \hangafter=1 \verb'sort' options, avoiding (\verb'sort'
选项, 避免) 91, 140

% \hangindent=2pc \hangafter=1 \marginpar{210}
\hangindent=2pc \hangafter=1 sort programs, testing (排序程序的测试) 155

\hangindent=2pc \hangafter=1 sort test program, batch (排序测试程序,
批处理模式) 155

\hangindent=2pc \hangafter=1 \verb'sortgen' language (\verb'sortgen' 语言) 140

\hangindent=2pc \hangafter=1 \verb'sortgen' program (程序 \verb'sortgen') 140

\hangindent=2pc \hangafter=1 sorting address list (排序地址列表) 84

\hangindent=2pc \hangafter=1 sorting dates (排序数据) 72

\hangindent=2pc \hangafter=1 sorting, topological (拓扑排序) 170

\hangindent=2pc \hangafter=1 \verb'split' function (函数 \verb'split')
41,52-53, 76, 80, 84, 192

\hangindent=2pc \hangafter=1 \verb'sprintf' function (函数 \verb'sprintf')
42, 76, 88, 189

\hangindent=2pc \hangafter=1 \verb'sqrt' function (函数 \verb'sqrt') 39

\hangindent=2pc \hangafter=1 \verb'srand' function (函数 \verb'srand') 40,111

\hangindent=2pc \hangafter=1 stack (栈) 142,200,204

\hangindent=2pc \hangafter=1 standard error file (标准错误文件) 59

\hangindent=2pc \hangafter=1 standard input file (标准输入文件) 59, 66

\hangindent=2pc \hangafter=1 standard input filename, \verb'-'
(标准输入的文件名) 64, 116

\hangindent=2pc \hangafter=1 standard output file (标准输出文件) 5, 56

\hangindent=2pc \hangafter=1 statement, \verb'break' (\verb'break' 语句) 49

\hangindent=2pc \hangafter=1 statement, \verb'close' (\verb'close' 语句) 59, 82

\hangindent=2pc \hangafter=1 statement, \verb'continue' (\verb'continue'
语句) 49

\hangindent=2pc \hangafter=1 statement, \verb'delete' (\verb'delete' 语句) 52

\hangindent=2pc \hangafter=1 statement, \verb'do' (\verb'do' 语句) 49

\hangindent=2pc \hangafter=1 statement, empty (空语句) 50, 188

\hangindent=2pc \hangafter=1 statement, \verb'exit' (\verb'exit' 语句) 49

\hangindent=2pc \hangafter=1 statement, \verb'for' (\verb'for' 语句) 16, 49

\hangindent=2pc \hangafter=1 statement, \verb'for' ... \verb'in'
(\verb'for' ... \verb'in' 语句) 51

\hangindent=2pc \hangafter=1 statement, \verb'if'-\verb'else'
(\verb'if'-\verb'else' 语句) 14, 47

\hangindent=2pc \hangafter=1 statement, \verb'next' (\verb'next' 语句) 49

\hangindent=2pc \hangafter=1 statement, pattern-action (\patact 语句)
iii, 2, 21, 34, 53, 187

\hangindent=2pc \hangafter=1 statement, \verb'print' (\verb'print' 语句) 5, 55

\hangindent=2pc \hangafter=1 statement, \verb'printf' (\verb'printf' 语句)
7, 24, 56, 98

\hangindent=2pc \hangafter=1 statement, \verb'return' (\verb'return' 语句) 53

\hangindent=2pc \hangafter=1 statement, \verb'while' (\verb'while' 语句) 15, 47

\hangindent=2pc \hangafter=1 statements, continuing long (延续长语句) 15, 22, 188

\hangindent=2pc \hangafter=1 statements, summary of control-flow
(流程控制语句的总结) 48

\hangindent=2pc \hangafter=1 statements, summary of output (输出语句总结) 55

\hangindent=2pc \hangafter=1 status return (返回状态) 50, 64

\hangindent=2pc \hangafter=1 string, coercion to (强制转换成字符串) 45

\hangindent=2pc \hangafter=1 string comparison (字符串比较) 25-26, 44, 184

\hangindent=2pc \hangafter=1 string concatenation (字符串拼接)
13, 40, 43,  47, 56, 101, 182, 184

\hangindent=2pc \hangafter=1 string constant, \verb'"'...\verb'"' (字符串常量)
7, 24, 35, 51

\hangindent=2pc \hangafter=1 string functions, summary of (字符串函数的总结) 190

\hangindent=2pc \hangafter=1 string functions, table of (字符串函数汇总,
表格) 42

\hangindent=2pc \hangafter=1 string, null (空字符串) 13, 24, 42, 114, 192

\hangindent=2pc \hangafter=1 string, numeric value of a (字符串的数值值) 45

\hangindent=2pc \hangafter=1 string or number (字符串还是数值) 44

\hangindent=2pc \hangafter=1 string to number coercion (字符串强制转换成数值) 25

\hangindent=2pc \hangafter=1 string to number conversion (字符串转换成数值)
35, 44, 192

\hangindent=2pc \hangafter=1 string variables (字符串变量) 12,44

\hangindent=2pc \hangafter=1 string-matching pattern (字符串匹配模式) 26

\hangindent=2pc \hangafter=1 string-matching patterns, summary of (字符串匹 
配模式的总结) 27

\hangindent=2pc \hangafter=1 strings as regular expressions
(字符串作为正则表达式) 40

\hangindent=2pc \hangafter=1 \verb'sub' function (函数 \verb'sub') 42, 182

\hangindent=2pc \hangafter=1 \verb'sub', side-effects of (\verb'sub'
的副作用) 43

\hangindent=2pc \hangafter=1 subscripts, array (数组的下标) 50-52

\hangindent=2pc \hangafter=1 subscripts, numeric (数值下标) 52

\hangindent=2pc \hangafter=1 \verb'SUBSEP' variable (变量 \verb'SUBSEP') 36, 53

\hangindent=2pc \hangafter=1 \verb'subset' function (函数 \verb'subset') 109

\hangindent=2pc \hangafter=1 substitution, \verb'&' in (字符串替换中的
\verb'&') 43, 72, 189

\hangindent=2pc \hangafter=1 \verb'substr' function (函数 \verb'substr') 43, 72

\hangindent=2pc \hangafter=1 substring (子字符串) 24

\hangindent=2pc \hangafter=1 successor node (后继节点) 170

\hangindent=2pc \hangafter=1 successor-list data structure (后继节点列表) 171

\hangindent=2pc \hangafter=1 \verb'suffix' function (函数 \verb'suffix') 105

\hangindent=2pc \hangafter=1 \verb'sum1' program (程序 \verb'sum1') 68

\hangindent=2pc \hangafter=1 \verb'sum2' program (程序 \verb'sum2') 68

\hangindent=2pc \hangafter=1 \verb'sum3' program (程序 \verb'sum3') 69

\hangindent=2pc \hangafter=1 \verb'sumcomma' program (程序 \verb'sumcomma') 71

\hangindent=2pc \hangafter=1 summary of actions (动作的总结) 34,188

\hangindent=2pc \hangafter=1 summary of arithmetic functions (算术函数总结) 190

\hangindent=2pc \hangafter=1 summary of control-flow statements
(流程控制语句的总结) 48

\hangindent=2pc \hangafter=1 summary of expressions (表达式的总结) 37

\hangindent=2pc \hangafter=1 summary of input-output (输入输出总结) 188

\hangindent=2pc \hangafter=1 summary of operators (运算符总结) 190

\hangindent=2pc \hangafter=1 summary of output statements (输出语句总结) 55

\hangindent=2pc \hangafter=1 summary of patterns (模式的总结) 23,187

\hangindent=2pc \hangafter=1 summary of \verb'printf' speciflcations
(\verb'printf' 格式说明符的总结) 189

\hangindent=2pc \hangafter=1 summary of regular expressions (正则表达式的总结)
28, 191

\hangindent=2pc \hangafter=1 summary of string functions (字符串函数的总结) 190

\hangindent=2pc \hangafter=1 summary of string-matching patterns (字符串匹配 
模式的总结) 27

\hangindent=2pc \hangafter=1 summing columns (列求和) 67

\hangindent=2pc \hangafter=1 Swartwout, D. vi

\hangindent=2pc \hangafter=1 symbol table (符号表) 131, 134, 152

\hangindent=2pc \hangafter=1 syntax error (语法错误) 4

\hangindent=2pc \hangafter=1 \verb'system' function (函数 \verb'system') 59, 64

\pdfbookmark[2]{T}{T}\medskip\textbf{\large{T}}

\hangindent=2pc \hangafter=1 tab character, \verb'\t' (制表符) 15, 24, 31

\hangindent=2pc \hangafter=1 table, base (基表) 106

\hangindent=2pc \hangafter=1 table, database (数据库的表) 103

\hangindent=2pc \hangafter=1 table, derived (导出表) 106

\hangindent=2pc \hangafter=1 table formatting (表格的格式化) 95

\hangindent=2pc \hangafter=1 table of arithmetic functions (算术函数表格) 39

\hangindent=2pc \hangafter=1 table of arithmetic operators (算术运算符表格) 46

\hangindent=2pc \hangafter=1 table of assembler instructions (汇编指令表格) 132

\hangindent=2pc \hangafter=1 table of built-in variables (内建变量表格) 36

\hangindent=2pc \hangafter=1 table of comparison operators (比较运算符表格) 25

\hangindent=2pc \hangafter=1 table of escape sequences (转义序列表格) 31

\hangindent=2pc \hangafter=1 table of \verb'getline' forms (\verb'getline'
的形式) 62

\hangindent=2pc \hangafter=1 table of patterns (模式表格) 33

\hangindent=2pc \hangafter=1 table of performance measurements (性能测试表格)
183

\hangindent=2pc \hangafter=1 table of \verb'printf' examples (\verb'printf'
的例子) 57

\hangindent=2pc \hangafter=1 table of \verb'printf' specifications
(\verb'printf' 的格式说明符) 57

\hangindent=2pc \hangafter=1 table of regular expressions (正则表达式表格) 32

\hangindent=2pc \hangafter=1 table of string functions (字符串函数表格) 42

\hangindent=2pc \hangafter=1 \verb'table' program (程序 \verb'table') 98

\hangindent=2pc \hangafter=1 table, symbol (符号表) 131,134,152

\hangindent=2pc \hangafter=1 \verb'table1' program (程序 \verb'table1') 196

\hangindent=2pc \hangafter=1 \verb'tbl' command (命令 \verb'tbl') 95

\hangindent=2pc \hangafter=1 terminal symbol (终结符) 113, 145

\hangindent=2pc \hangafter=1 test framework program (测试框架程序) 159

\hangindent=2pc \hangafter=1 test program, interactive (交互式的测试程序) 157

\hangindent=2pc \hangafter=1 test, side-effects of (条件判断的副作用) 52, 192

\hangindent=2pc \hangafter=1 testing, boundary condition (边界条件测试) 155

\hangindent=2pc \hangafter=1 testing, interactive (交互式的测试) 156

\hangindent=2pc \hangafter=1 testing sort programs (测试排序程序) 155

\hangindent=2pc \hangafter=1 TEX formatter (排版程序 TEX) 120,124

\hangindent=2pc \hangafter=1 text justification (测试验证) 98, 201

\hangindent=2pc \hangafter=1 timing tests (计时测试) 183

\hangindent=2pc \hangafter=1 Toolchest (工具箱) vi

\hangindent=2pc \hangafter=1 topological sort algorithm (拓扑排序算法) 171

\hangindent=2pc \hangafter=1 topological sorting (拓扑排序) 170

\hangindent=2pc \hangafter=1 \verb'tr' command (命令 \verb'tr') 201

\hangindent=2pc \hangafter=1 translator model (翻译器模型) 131

\hangindent=2pc \hangafter=1 \verb'transpose' program (程序
\verb'transpose') 204

\hangindent=2pc \hangafter=1 tree, binary (二叉树) 163

\hangindent=2pc \hangafter=1 Trickey, H. W. vi

\hangindent=2pc \hangafter=1 \verb'troff' command (命令 \verb'troff')
120,124-125, 127, 139

\hangindent=2pc \hangafter=1 \verb'tsort' program (程序 \verb'tsort') 172

\pdfbookmark[2]{U}{U}\medskip\textbf{\large{U}}

\hangindent=2pc \hangafter=1 Ullman, J. D. 110,152, 179

\hangindent=2pc \hangafter=1 \verb'unbundle' program (程序 \verb'unbundle') 82

\hangindent=2pc \hangafter=1 underscore, \verb'-' (下划线) 35

\hangindent=2pc \hangafter=1 \verb'unget' function (函数 \verb'unget') 105

\hangindent=2pc \hangafter=1 uninitialized variables (未初始化的变量) 51, 58

\hangindent=2pc \hangafter=1 universal relation (全局关系表) 107

\hangindent=2pc \hangafter=1 update algorithm, \verb'make' (\verb'make'
的更新算法) 176

\hangindent=2pc \hangafter=1 updating, file (文件的更新) 175

\hangindent=2pc \hangafter=1 user-defined functions (用户自定义函数) 
53, 182,  187

\pdfbookmark[2]{V}{V}\medskip\textbf{\large{V}}

\hangindent=2pc \hangafter=1 value of a string, numeric (字符串的数值值) 45

\hangindent=2pc \hangafter=1 value of comparison expression (比较表达式的值) 37

\hangindent=2pc \hangafter=1 value of logical expression (逻辑表达式的值) 37

\hangindent=2pc \hangafter=1 van Eick. P. vi

\hangindent=2pc \hangafter=1 VanWyk,C.J. vi

\hangindent=2pc \hangafter=1 variable, \verb'$0' record (变量, \verb'$0') 5, 35

\hangindent=2pc \hangafter=1 variable, \verb'ARGC' (变量 \verb'ARGC')
36, 63, 189

\hangindent=2pc \hangafter=1 variable, \verb'ARGV' (变量 \verb'ARGV')
36, 63-65, 116, 189

\hangindent=2pc \hangafter=1 variable assignment, command line
(命令行的变量赋值) 63

\hangindent=2pc \hangafter=1 variable, \verb'FILENAME' (变量
\verb'FILENAME') 33, 35-36, 81, 103

\hangindent=2pc \hangafter=1 variable, \verb'FNR' (变量 \verb'FNR') 33, 35-36, 61

\hangindent=2pc \hangafter=1 variable, \verb'FS' (变量 \verb'FS')
24, 35-36, 52, 60, 83, 135, 187

\hangindent=2pc \hangafter=1 variable names, rules for (变量的取名规则) 35

\hangindent=2pc \hangafter=1 variable, \verb'NF' (变量 \verb'NF')
6, 14, 35-36, 61

\hangindent=2pc \hangafter=1 variable, \verb'NR' (变量 \verb'NR')
6, 12, 14, 35-36, 61

\hangindent=2pc \hangafter=1 variable, \verb'OFMT' (变量 \verb'OFMT') 36, 45

\hangindent=2pc \hangafter=1 variable, \verb'OFS' (变量 \verb'OFS')
35-36, 43, 55-56

\hangindent=2pc \hangafter=1 variable, \verb'ORS' (变量 \verb'ORS') 36, 55-56, 83

\hangindent=2pc \hangafter=1 variable, \verb'RLENGTH' (变量 \verb'RLENGTH')
35-36, 41

\hangindent=2pc \hangafter=1 variable, \verb'RS' (变量 \verb'RS') 36, 60, 83-84

\hangindent=2pc \hangafter=1 variable, \verb'RSTART' (变量 \verb'RSTART')
35-36, 41, 189

\hangindent=2pc \hangafter=1 variable, \verb'SUBSEP' (变量 \verb'SUBSEP') 36, 53

\hangindent=2pc \hangafter=1 variables, field (字段变量) 35

\hangindent=2pc \hangafter=1 variables, global (全局变量) 54,116

\hangindent=2pc \hangafter=1 variables, local (局部变量) 54, 116, 182

\hangindent=2pc \hangafter=1 variables, numeric (数值变量) 44

\hangindent=2pc \hangafter=1 variables, string (字符串变量) 12,44

\hangindent=2pc \hangafter=1 variables, table of built-in (内建变量表格) 36

\hangindent=2pc \hangafter=1 variables, uninitialized (未初始化的变量) 51, 58

\pdfbookmark[2]{W}{W}\medskip\textbf{\large{W}}

\hangindent=2pc \hangafter=1 \verb'wc' command (命令 \verb'wc') 183

\hangindent=2pc \hangafter=1 \verb'while' statement (\verb'while' 语句) 15,47

\hangindent=2pc \hangafter=1 \verb'who' command (变量 \verb'who') 62

\hangindent=2pc \hangafter=1 wild-card characters (通配符) 26

\hangindent=2pc \hangafter=1 Williams, J. W. J. 162

\hangindent=2pc \hangafter=1 word count program (单词计数程序) 14, 119

\hangindent=2pc \hangafter=1 \verb'wordfreq' program (程序 \verb'wordfreq') 119

\hangindent=2pc \hangafter=1 words, convert numbers to (把数字转化成单词) 76

\pdfbookmark[2]{X}{X}\medskip\textbf{\large{X}}

\hangindent=2pc \hangafter=1 \verb'xref' program (程序 \verb'xref') 122

\pdfbookmark[2]{Y}{Y}\medskip\textbf{\large{Y}}

\hangindent=2pc \hangafter=1 \verb'yacc' parser generator
(语法分析器生成程序 \verb'yacc') 152,175

\hangindent=2pc \hangafter=1 Yannakakis, M. vi

\end{multicols}
